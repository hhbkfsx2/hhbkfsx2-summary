\section{LF04 - Einfache IT-Systeme (Wissmann)}


%%% Anfang: tl;dr
\subsection{tl;dr - Zusammenfassung der Zusammenfassung}
%%% Ende: tl;dr
%%%%%%%%%%%%%%%%%%%%%%%%%%%%%%%%%%%%%%%%%%%%%%%%%%%%%%%%%%%%%%%%%%%%%%%%%%%%%%%%

\subsection{Elektrische Grundgrößen}
	\subsubsection{Elektrische Ladung, Spannung und Potential}
		\paragraph{Elementarladung}~\\
		\paragraph{Elektrische Ladung}~\\
		\paragraph{Entstehen einer elektrischen Ladung}~\\
		\paragraph{Definition: elektrische Spannung}~\\
		\paragraph{Spannungsmessung}~\\
		\paragraph{Elektrisches Potential}~\\
	\subsubsection{Spannungsarten}
		\paragraph{Gleichspannung}~\\
		\paragraph{Wechselspannung}~\\
		\paragraph{Mischspannung}~\\
		\paragraph{Kenngrößen der Netzwechselspannung}~\\
	\subsubsection{Elektrischer Strom und Stromdichte}
		\paragraph{Modellvorstellung}~\\
		\paragraph{Elektrischer Stromkreis}~\\
		\paragraph{Stromgeschwindigkeit}~\\
		\paragraph{Elektrische Stromstärke}~\\
		\paragraph{Messung der Stromstärke}~\\
		\paragraph{Stromwirkung}~\\
		\paragraph{Elektrische Stromdichte}~\\
	\subsubsection{Elektrischer Widerstand und Leitwert}
	\subsubsection{Ohmsches Gesetz}
	\subsubsection{Elektrischer Widerstand von Leitern}
	\subsubsection{Spannungsabfall auf Leitern}
	\subsubsection{Elektrische Leistung}
		\paragraph{Definition}~\\
		\paragraph{Nennleistung}~\\
		\paragraph{Messung der elektrische Leistung}~\\
	\subsubsection{Elektrische Arbeit}
	\subsubsection{Messung der elektrische Leistung mittels Elektrizitätszähler}
	\subsubsection{Wirkungsgrad}
	
%%% Ende: Elektrische Grundgrößen

\subsection{Zusammenschaltung von Widerständen}
\subsubsection{Reihenschaltung}
\subsubsection{Parallelschaltung}
\subsubsection{Gemischte Schaltungen}
\subsubsection{Spannungsleiter}
\subsubsection{Arten von Widerständen}
%%% Ende: Zusammenschaltung von Widerständen

\subsection{Kondensatoren und elektrisches Feld}
\subsubsection{Elektrisches Feld eines Kondensators}
\subsubsection{Kondensatoren als Ladungsspeicher}
\subsubsection{Schaltungen von Kondensatoren}
\subsubsection{Kondensatoren im Gleichstromkreis}
%%% Ende: Kondensatoren und elekrisches Feld

\subsection{Spule und magnetisches Feld}
\subsubsection{Magnetisches Feld in einer Spule}
\subsection{Spule im Gleichstromkreis}
%%% Ende: Spule und magnetisches Feld

\subsection{Elektromagnetische Verträglichkeit}

%%% Ende: Elektromagnetische Verträglichkeit
