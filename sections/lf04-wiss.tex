\section{LF04 - Einfache IT-Systeme (Wissmann)}


%%% Anfang: tl;dr
\subsection{tl;dr - Zusammenfassung der Zusammenfassung}

%%% Anfang: tl;dr > Definitionen
\subsubsection{Definitionen}
\begin{tabular}{	p{\dimexpr 0.4\linewidth-2\tabcolsep}
				p{\dimexpr 0.6\linewidth-2\tabcolsep}}
Elektr. Ladung & Eine Menge von Elementarladungenen nennt man elektrische Ladung\\
Elektr. Spannung & Die elektrische Spannung ist das Ausgleichsbestreben getrennter elektrischer Ladung\\
Elektr. Potential & Ein elektrisches Potential ist eine Spannungsangabe gegenüber einem Bezugspunkt (meinst Ground)\\
Elektr. Strom & Ein elektrischer Strom ist die gerichtete Bewegung von Ladungsträgern (im Leiter = Elektronen)\\
Elektr. Stromstärke & Die elektrische Stromstärke ist die Ladungsmenge, die pro Sekunde durch den Leitungsquerschnitt fließt\\
Elektr. Stromdichte & Die Stromdichte gibt den Strom pro Flächeneinheit an und ermöglicht die Beurteilung von bspw. der Erwärmung des Leiters\\
Elektr. Widerstand & Elektrischer Widerstand ist die Eigenschaft eines Leiters die Fortbewegung elektrischer Ladungsträger zu behindern\\
\end{tabular}

%%% Anfang: tl;dr > Formelzeichen
\subsubsection{Formelzeichen und Einheiten}
\begin{tabular}{llll}
{\bf Begriff}		& {\bf Formezeichen} & {\bf Einheit} & {\bf Wert}\\
Elektr. Ladung		& $Q$ & $C$ & $6.24 \times 10^{19} \times e$\\
Elektr. Spannung		& $U$ & $V$ & $1\frac{Nm}{C}$\\
Elektr. Potential	& $\phi$ (Phi) & $V$ & \\
Elektr. Stromstärke	& $I$ & $A$ & \\
Elektr. Stromdichte	& $J$ & $1\frac{A}{mm^2}$ & \\
Elektr. Widerstand	& $R$ & $\Omega$ & \\
Spez. Widerstand		& $\varrho$ (Rho) & $1\frac{\Omega mm^2}{m}$ & \\
Spez. Leitfähigkeit	& $\gamma$ & $1\frac{m}{\Omega mm^2}$ & \\
\end{tabular}

%%% Anfang: tl;dr > Formeln
\subsubsection{Formeln}
\begin{tabular}{llll}
Elektr. Stromstärke	& $I$ & $=$ & $\frac{Q}{t}$\\
Elektr. Stromdichte	& $J$ & $=$ & $\frac{I}{A}$\\
Elektr. Widerstand 	& $U$ & $=$ & $R\times I$\\
Elektr. Widerstand	& $R$ & $=$ & $\varrho\times\frac{l}{A}$\\
Spez. Widerstand		& $\varrho$	& $=$ & $\frac{AR}{l}$\\
Länge des Leiters	& $l$		& $=$ & $\frac{AR}{\varrho}$\\
Querschnitt			& $A$		& $=$ & $\varrho\times\frac{l}{R}$\\
Leitfähigkeit		& $\gamma$	& $=$ & $\frac{1}{\varrho} = 1\frac{m}{\Omega mm^2} = 1\frac{10^6}{\Omega} = \frac{1\times 10^6\times S}{m} = 1\frac{MS}{m}$\\
Leitfähigkeit		& $R_{LTG}$ 	& $=$ & $\frac{l}{A\gamma}$\\
\end{tabular}

%%% Ende: tl;dr
%%%%%%%%%%%%%%%%%%%%%%%%%%%%%%%%%%%%%%%%%%%%%%%%%%%%%%%%%%%%%%%%%%%%%%%%%%%%%%%%

\subsection{Elektrische Grundgrößen}
	\subsubsection{Elektrische Ladung, Spannung und Potential}
		\paragraph{Elementarladung}~\\
		
\begin{tabular}{llll}
Proton:		& positive Elementarladung &	$e^+$ & \\
Elektron:	& negative Elementarladung &	$e^-$ & \\
& &	& $e = 1.6 \times 10^{-19} As$ \\
\end{tabular}

		\paragraph{Elektrische Ladung}~\\
		
\noindent Eine Menge von Elementarladungen nennt man elektrische Ladung.\\\\
\begin{tabular}{lllll}
Formelzeichen	& $=$ & $Q$ & & \\
Einheit			& $=$ & $C$ & $=$ & $6.24 \times 10^{19} \times e$
\end{tabular}	
		
		\paragraph{Entstehung von Spannung}~\\
		
\noindent Elektrische Spannung entsteht, wenn durch Arbeitsaufwand Ladungen getrennt werden. Es bedarf einer Kraft, um $U$ zu überwinden.
		
		\paragraph{Definition: elektrische Spannung}~\\

\noindent Die elektrische Spannung ist das Ausgleichsstreben getrennter elektrischer Ladung.

\begin{tabular}{lllll}
Formelzeichen	& $=$ & $U$ & & \\
Einheit			& $=$ & $V$ & $=$ & $1\frac{Nm}{C}$ \\
\end{tabular}	
		
		\paragraph{Spannungsmessung}~\\
		\paragraph{Elektrisches Potential}~\\
		
\noindent Ein elektrisches Potential ist eine Spannungsangabe gegenüber einem Bezugspunkt\\(meistens: Masse [GND]).

\begin{tabular}{lll}
Formelzeichen	& $=$ & $\phi$ \\
Einheit			& $=$ & $V$ \\
\end{tabular}	
		
	\subsubsection{Spannungsarten}
		\paragraph{Gleichspannung}~\\
		\paragraph{Wechselspannung}~\\
		\paragraph{Mischspannung}~\\
		\paragraph{Kenngrößen der Netzwechselspannung}~\\
		
\begin{tabular}{lllll}
{\bf Kenngröße}	& {\bf Formelzeichen}	& {\bf Einheit} & {\bf Zahlwert} & {\bf Bemerkung}\\ 
Augenblickswert	& $U$ 					& $1V$	& & $u(t)=\hat{u}\times sin(2\pi f t)$\\
Scheitelwert		& $\widehat{U}$			& $1V$	& $325V$		& größter Wert der Spannung\\
Spitze-Spitze	& $U_{ss}$				& $1V$	& $650V$		& \\
Effektivwert		& $U_{eff}$				& $1V$	& $230V$		& $U_{eff}=\frac{\hat{u}}{\sqrt{2}}$\\
Periodendauer	& $r$					& $1s$	& $0.02s$	& \\
Frequenz			& $f$					& $1$	& $50 Hz$	& $T = \frac{1}{f}$ \\
\end{tabular}		
		
	\subsubsection{Elektrischer Strom und Stromdichte}
		\paragraph{Modellvorstellung}~\\

\noindent Elektrischer Strom ist die gerichtete Bewegung von Ladungsträgern (im Leiter = Elektronen).

		\paragraph{Elektrischer Stromkreis}~\\
		
\noindent Elektronen fließen vom $-$Pol zum $+$Pol. Die technische Stromrichtung ist allerdings umgekehrt: $+ \to -$.		
		
		\paragraph{Stromgeschwindigkeit}~\\
		
\noindent Die Geschwindigkeit von Elektronen beträgt etwa $0.001mm$ bis $10mm$ pro Sekunde. Bei $1A$ beträgt die Elektronengeschwindigkeit etwa $1\frac{mm}{s}$. Im Gegensatz dazu beträgt die Signalausbreitungsgeschwindigkeit typischerweise etwas mehr als die halbe Lichtgeschwindigkeit ($0.6\times c$).
	
		\paragraph{Elektrische Stromstärke}~\\
		
\noindent Die elektrische Stromstärke ist die Ladungsmenge, die pro Sekunde durch den Leitungsquerschnitt fließt.

\begin{tabular}{llll}
Formelzeichen	& $=$ & $I$ &\\
Einheit			& $=$ & $A$ &\\
& & & $I = \frac{Q}{t}$\\
\end{tabular}	
		
		\paragraph{Messung der Stromstärke}~\\
		\paragraph{Stromwirkung}~\\
		
\noindent Lichtwirkung, Wärmewirkung, magnetische Wirkung, chemische Wirkung, physiologische Wirkung\dots		
		
		\paragraph{Elektrische Stromdichte}~\\
		
\noindent Die Stromdichte gibt den Strom pro Flächeneinheit an ermöglicht die Beurteilung von beispielsweise die Erwärmung des Leiters.	

\begin{tabular}{llll}
Formelzeichen	& $=$ & $J$ &\\
Einheit			& $=$ & $1\frac{A}{mm^2}$ &\\
\end{tabular}\newline

\begin{tabular}{llll}
Beispiel: & Lampe & $55W$ bei $12V$ $\widehat{=}\ 4.5A$ &\\
& & $A_{LTG} = 1.5mm^2$ & $= 3\frac{A}{mm^2}$\\
& & $A_{Lampe} = 0.006mm^2$ & $= 750\frac{A}{mm^2}$\\
\end{tabular}
		
	\subsubsection{Elektrischer Widerstand und Leitwert}
		\paragraph{Definition}~\\

\noindent Elektrischer Widerstand ist die Eigenschaft eines Leiters die Fortbewegung elektrischer Ladungsträger zu behindern.

		\paragraph{Ohmsches Gesetz}~\\
		
\noindent Bei einem elektrischen Widerstand ist die Stromstärke proportional zu der Spannung ($I \sim U$) und umgekehrt proportional zum Widerstand ($I \sim \frac{1}{R}$).

		\paragraph{Elektrischer Widerstand von Leitern}~\\
		
\noindent Der Widerstand ist proportional zur Länge des Leiters ($R \sim l$), umgekehrt proportional zum Querschnitt ($R \sim \frac{1}{A}$) und abhängig vom Material. Mit dem Faktor $\varrho$ wird die Materialabhängigkeit berücksichtigt. $\varrho$ ist der spezifische Widerstand. Leitfähigkeit bezeichnet den Kehrwert des spezifischen Widerstandes ($\gamma$ oder $\kappa$).\\

\begin{tabular}{llll}
Formelzeichen	& $=$ & $\varrho$ &\\
Einheit			& $=$ & $1\frac{\Omega mm^2}{m	}$ &\\
\end{tabular}\newline

\begin{tabular}{lll}
$R$			& $=$ & $\varrho\times\frac{l}{A}$\\
$\varrho$	& $=$ & $\frac{AR}{l}$\\
$l$			& $=$ & $\frac{AR}{\varrho}$\\
$A$			& $=$ & $\varrho\times\frac{l}{R}$\\
$\gamma$		& $=$ & $\frac{1}{\varrho} = 1\frac{m}{\Omega mm^2} = 1\frac{10^6}{\Omega} = \frac{1\times 10^6\times S}{m} = 1\frac{MS}{m}$
\end{tabular}

		\paragraph{Spannungsabfall auf Leitern}~\\
		
\noindent Ein Motor soll über eine $100m$ lange Leitung angeschlossen werden. Dabei fließt ein Strom von $I = 16A$ bei einer Speisespannung von $U = 230V$. Der Querschnitt der Leitung beträgt $A_{LTG} = 1.5mm^2$ (Kuper). Gesucht ist die Spannung, die am Motor ankommt ($U_{Motor}$).\\

\begin{tabular}{lllllll}
$R_{LTG}$ & $=$ & $\frac{l}{A\times\gamma}$ & $=$ & $\frac{100m}{1.5mm^2\times 58\frac{m}{mm^2\times\Omega}}$ & $=$ & $1.15\Omega$\\
$U_{LTG}$ & $=$ & $R_{LTG}\times I$ & $=$ & $230V \times 16A$ & $=$ & $18.4V$\\
$U_{Motor}$ & $=$ & $U - 2\times U_{LTG}$ & $=$ & $230V - 2\times 18.4V$ & $=$ & $193.2V$ \\
\end{tabular}

\subsubsection{Elektrische Leistung}

\paragraph{Definition}~\\
%Verbraucher, die Strom in Wärme umwandeln (Ohmsche Verbraucher)
Die Leistung ist das Produkt aus Spannung und Strom.
\begin{tabular}{lll}
Formelzeichen & $P$ &\\
Einheit & $W$ & $1W = 1V\times 1A$
& & $P = U\times I \widehat{=} R^2\times I \widehat{=} \frac{U^2}{R}$\\
\end{tabular}

\paragraph{Nennleistung}~\\
Die Nennleistung gibt an, welche Leistung dauernd (z.B. an der Motorwelle) abgegeben werden kann. Über die zugeführte Leistung gibt der Wirkungsgrad Auskunft.

\paragraph{Wirkungsgrad}~\\
Der Wirkungsgrad $\eta$ (eta) gibt an, wie viel Prozent der zugeführten Leistung in nutzbare Leistung umgewandelt werden. $\eta = \frac{P_{ab}}{P_{zu}}}\times 100$
Werden Anlagen im Verbund betrieben, muss der Gesamtwirkungsgrad berechnet werden.
$\eta_{ges} = \frac{P_4}{P_1}$\\
Bsp.: $\eta_1 = 40%$, $\eta_2 = 60%$, $\eta_3 = 90%$\\
$\eta_{ges} = \eta_1\times\eta_2\times\eta_3\dots$
Bsp.: $\eta_{ges} = 0.4\times0.6\times0.9 = 21.6%$

\paragraph{Messung der elektrische Leistung}~\\
a) Indirekte Methode: Messung von Strom und Spannung mit anschließender Multiplikation. b) Direkte Methode: dabei wirkt ein Spannungsmesswerk und ein Strommesswerk direkt auf einen Zeiger

%%% Anfang: > Elektrische Arbeit
\subsubsection{Elektrische Arbeit}

\paragraph{Definition}~\\
Elektrische Arbeit wird verrichtet, wenn ein Verbraucher mit der elektrischen Leistung $P$ eine bestimmte Zeit $t$ eingesetzt wird.
\begin{tabular}{lll}
Formelzeichen & $W$ &\\
Einheit & $1VAs \widehat{=} 1Ws$ (elektrisch) &\\
& & $W = P\times t$\\
\end{tabular}
%1Ws = 1Nm (mech) = 1J (therm)

\paragraph{Kosten der elektrische Arbeit}~\\
Bsp.: Glühlampe $P = 100W$
\begin{tabular}{lll}

\end{tabular}



\subsubsection{Messung der elektrische Leistung mittels Elektrizitätszähler}
	
%%% Ende: Elektrische Grundgrößen

\subsection{Zusammenschaltung von Widerständen}
\subsubsection{Reihenschaltung}
\subsubsection{Parallelschaltung}
\subsubsection{Gemischte Schaltungen}
\subsubsection{Spannungsleiter}
\subsubsection{Arten von Widerständen}
%%% Ende: Zusammenschaltung von Widerständen

\subsection{Kondensatoren und elektrisches Feld}
\subsubsection{Elektrisches Feld eines Kondensators}
\subsubsection{Kondensatoren als Ladungsspeicher}
\subsubsection{Schaltungen von Kondensatoren}
\subsubsection{Kondensatoren im Gleichstromkreis}
%%% Ende: Kondensatoren und elekrisches Feld

\subsection{Spule und magnetisches Feld}
\subsubsection{Magnetisches Feld in einer Spule}
\subsubsection{Spule im Gleichstromkreis}
%%% Ende: Spule und magnetisches Feld

\subsection{Elektromagnetische Verträglichkeit}

%%% Ende: Elektromagnetische Verträglichkeit
