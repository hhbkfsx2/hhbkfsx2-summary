\section{Politik und Gesellschaftslehre}

%%% Anfang: 
\subsection{tl;dr - Zusammenfassung der Zusammenfassung}
%%% Ende:
%%%%%%%%%%%%%%%%%%%%%%%%%%%%%%%%%%%%%%%%%%%%%%%%%%%%%%%%%%%%%%%%%%%%%%%%%%%%%%%%

%%% Anfang: Lernen
\subsection{Lebenslanges Lernen}

%%% Ende: Lernen
%%%%%%%%%%%%%%%%%%%%%%%%%%%%%%%%%%%%%%%%%%%%%%%%%%%%%%%%%%%%%%%%%%%%%%%%%%%%%%%%

%%% Anfang: Personalentwicklung
\subsection{Personalentwicklung - Definition}

%%% Anfang: Personalentwicklung > Prinzipien
\subsubsection{Prinzipien einer zukunftsorientierten Personalentwicklung}

%%% Anfang: Personalentwicklung > Personalentwicklung
\subsubsection{Personalentwicklung}

%%% Anfang: Personalentwicklung > Adressaten
\subsubsection{Adressaten der Personalentwicklung}

%%% Ende: Personalentwicklung
%%%%%%%%%%%%%%%%%%%%%%%%%%%%%%%%%%%%%%%%%%%%%%%%%%%%%%%%%%%%%%%%%%%%%%%%%%%%%%%%

%%% Anfang: Rente/Alterarmut
\subsection{Rente und Altersarmut}

Der Begriff {\it Demographischer Wandel} bezeichnet, auf Deutschland angewendet, den wachsenden Altersdurchschnitt der Bevölkerung. Ein Faktor des demographischen Wandels ist, dass weniger Kinder geboren werden und weniger netto Einwanderung besteht als für den Erhalt der Bevölkerungsgröße notwendig wären. Dadurch nimmt die Zahl der älteren Menschen und damit auch die Zahl der Rentner stetig zu.

Der Generationenvertrag wird durch diese Entwicklung in Frage gestellt.

%%% Ende: Rente/Altersarmut
%%%%%%%%%%%%%%%%%%%%%%%%%%%%%%%%%%%%%%%%%%%%%%%%%%%%%%%%%%%%%%%%%%%%%%%%%%%%%%%%