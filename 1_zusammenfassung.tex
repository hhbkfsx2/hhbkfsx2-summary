\documentclass[11pt,a4paper,oneside,ngerman]{article}

\usepackage[left=2.5cm,right=2.5cm,top=2.5cm,bottom=2.5cm]{geometry}

\usepackage{lmodern}									% Latin Modern
\renewcommand*\familydefault{\sfdefault} 			% Only if the base font of the document is to be sans serif
\usepackage[T1]{fontenc}

\usepackage[ngerman]{babel}							% Setzt die deutsche Sprache, sodass statt table of contents
													% Inhaltsverzeichnis geschrieben wird
\usepackage[utf8]{inputenc}    						% Koderiung: UTF-8
\usepackage{color}									% Ermöglicht farbigen Text: \textcolor{declared-color}{text}
													% Bsp.: \textcolor{red}{Dieser Text ist rot}	
\usepackage{graphicx}					
\usepackage{wrapfig}									% Ermöglicht Bildumlauf mit der wrapfigure-Umgebung
\usepackage{mathtools}
\usepackage{listings}								% Erlaut das einfügen von Quell-Code
										
\usepackage{fancyhdr} 
%\pagestyle{fancy}
\fancyhf{} 											% alle Kopf- und Fußzeilenfelder bereinigen;
													% löscht doppelte Seitenzahlen!
%\fancyhead[L]{} 									% Kopfzeile links
%\fancyhead[C]{}										% zentrierte Kopfzeile
%\fancyhead[R]{} 									% Kopfzeile rechts
\fancyfoot[R]{\thepage}				 				% Seitennummer

%\renewcommand{\headrulewidth}{0.4pt} %obere Trennlinie
%\renewcommand{\footrulewidth}{0.4pt} %untere Trennlinie			

\newcommand{\qr}{\textquotedblleft}				% Definiert eine Abkürzung für dt. Anführungsstriche links
\newcommand{\ql}{\quotedblbase}					% Definiert eine Abkürzung für dt. Anführungsstriche rechts

%%% Generelle Inforationen über das Dokument

\author{Kilian Engelhardt}
\title{Zusammenfassung d. Berufsschulunterrichts}
\date{\today}

\usepackage{hyperref}						% Weitere Optionen unter:
											% http://de.wikibooks.org/wiki/LaTeX-W%C3%B6rterbuch:_hyperref
\hypersetup{									% Hier werden Informationen für das PDF gesetzt
pdftitle={},
pdfauthor={},
pdfsubject={},
pdfkeywords={},
colorlinks=true,							
linkcolor=black								% Definiert die Farbe der Links vom Inhaltsverzeichnis
}											% zu den Sections im Dokument

%%% Beginn des Inhalts

\begin{document}

	\begin{center}
		\Huge{Zusammenfassung: Jahr 1}
	\end{center}

%\pagestyle{empty}	
%\include{zusammenfassung-deckblatt}
\tableofcontents
\newpage
\pagestyle{fancy}
\setcounter{page}{1}

% Die Lernfelder werden hinzugefügt 
%\include{sections/lf1a}
%\include{sections/lf1b}
%\include{sections/lf2}
%\include{sections/lf4-oeni.waec}
%\section{LF04 - Einfach IT-Systeme (Wiegand)}
%
\subsection{Einführung}
\subsection{CPU}
\subsection{Bussysteme}
\subsection{Halbleiterspeicher}
\subsection{Festplatte}
\subsection{BIOS}
\subsection{PC Sicherheit}

\section{LF04 - Einfache IT-Systeme (Wissmann)}
%
\subsection{Elektrische Grundgrößen}
\subsection{Zusammenschaltung von Widerständen}
\subsection{Kondensatoren und elektrisches Feld}
\subsection{Spule und magnetisches Feld}
\subsection{Elektromagnetische Verträglichkeit}

%\section{LF04 - Digitaltechnik}
%
\subsection{Zahlensysteme}

\paragraph{Umrechnung von Binär- in Hexadezimal- und Dezimal-Systemen}~\\
\paragraph{Binäre Darstellung von negativen Zahlen}~\\


\subsection{Codes}

\paragraph{ASCII}
Es gibt vom ASCII mehrere Abwandlungen, bei der einige wenig genutzte Zeichen durch regionale Sonderzeichen ersetzt werden. Beispielsweise handelt es sich bei ISO 636 (DIN 66003) um eine deutsche ASCII-Codierung, die auch Umlaute enthält.

\subsection{Schaltalgebra}
\subsection{Digitale Rechenschaltungen}
%\section{Lernfeld 5 - Fachliches Englisch}

%\section{Lernfeld 6 - Programmieren}

%%% Anfang: LS01
\subsection{LS01 -- Einführung in HTML und PHP}


%%% Ende: LS01

%%%%%%%%%%%%%%%%%%%%%%%%%%%%%%%%%%%%%%%%%%%%%%%%%%%%%%%%%%%%%%%%%%%%%%%%%%%%%%%%

%%% Anfang: LS02
\subsection{LS02 -- Einführung in Verzweigungen}

\paragraph{If-Anweisungen}
\paragraph{Switch-Case}
\begin{tabular}{l|l|l}

\end{tabular}
\lstinputlisting
	[caption={Ein Beispiel für Switch-Case-Anweisungen}
	\label{lst:Switch-Case},captionpos=b,language=PHP]
	{code/switch-case.php}
%%% Ende: LS02

%\include{sections/lf6db}
%\section{Deutsch und Kommunikation}

\subsection{Lernen}

\subsubsection{Physiologische Voraussetzungen des Lernerfolges}

\subsubsection{Äußere Einflussfaktoren auf den Lernerfolg}
\paragraph{Einfluss der direkten Umgebung}~\\
\paragraph{Einfluss des sozialen Umfeldes}~\\
\paragraph{Einfluss der Ernährung}~\\
\paragraph{Einfluss von Drogen}~\\

%\section{Politik und Gesellschaftslehre}

%%% Anfang: 
\subsection{tl;dr - Zusammenfassung der Zusammenfassung}
%%% Ende:
%%%%%%%%%%%%%%%%%%%%%%%%%%%%%%%%%%%%%%%%%%%%%%%%%%%%%%%%%%%%%%%%%%%%%%%%%%%%%%%%

%%% Anfang: Lernen
\subsection{Lebenslanges Lernen}

%%% Ende: Lernen
%%%%%%%%%%%%%%%%%%%%%%%%%%%%%%%%%%%%%%%%%%%%%%%%%%%%%%%%%%%%%%%%%%%%%%%%%%%%%%%%

%%% Anfang: Personalentwicklung
\subsection{Personalentwicklung - Definition}

%%% Anfang: Personalentwicklung > Prinzipien
\subsubsection{Prinzipien einer zukunftsorientierten Personalentwicklung}

%%% Anfang: Personalentwicklung > Personalentwicklung
\subsubsection{Personalentwicklung}

%%% Anfang: Personalentwicklung > Adressaten
\subsubsection{Adressaten der Personalentwicklung}

%%% Ende: Personalentwicklung
%%%%%%%%%%%%%%%%%%%%%%%%%%%%%%%%%%%%%%%%%%%%%%%%%%%%%%%%%%%%%%%%%%%%%%%%%%%%%%%%

%%% Anfang: Rente/Alterarmut
\subsection{Rente und Altersarmut}

Der Begriff {\it Demographischer Wandel} bezeichnet, auf Deutschland angewendet, den wachsenden Altersdurchschnitt der Bevölkerung. Ein Faktor des demographischen Wandels ist, dass weniger Kinder geboren werden und weniger netto Einwanderung besteht als für den Erhalt der Bevölkerungsgröße notwendig wären. Dadurch nimmt die Zahl der älteren Menschen und damit auch die Zahl der Rentner stetig zu.

Der Generationenvertrag wird durch diese Entwicklung in Frage gestellt.

%%% Ende: Rente/Altersarmut
%%%%%%%%%%%%%%%%%%%%%%%%%%%%%%%%%%%%%%%%%%%%%%%%%%%%%%%%%%%%%%%%%%%%%%%%%%%%%%%%
%\section{Credits}
Im Folgenden sind alle\footnote{Wer sich trotz eines Beitrages hier nicht wiederfindet, spricht mich am besten in der Schule darauf an.} Beitragenden zur Zusammenfassung aufgelistet:

\begin{enumerate}
	\item LF01 Gottwald:\\
Tobias Krenz
	\item LF02 Trenkmann:\\
Tobias Krenz
	\item LF04 Wiegand
	\item LF04 Oenings \& Wächter
	\begin{enumerate}
		\item Interrupts:
		\item Prozessmanagment:
		\item Lizenzen:
		\item Boot-Prozess:\\
Sebastian Heinke, Tobias Krenz, Jonathan Reuter
		\item Memory Managment:\\
Mirko Großmann
		\item OS: Windows
	\end{enumerate}
	\item LF04 Wissmann
	\item LF04 Digitaltechnik
	\item LF05 Wächter
	\item LF06 Abu Shebika
	\item LF06 Dresen
	\item DKO Fischer
	\begin{itemize}
		\item Lernmethoden:\\
Christian Flügel, David Piechaczek
		\item Physiologische Voraussetzungen des Lernerfolgs:\\
		Tobias Krenz
	\end{itemize}
	\item PK Trenkmann
	\item Korrekturgelesen von:
\end{enumerate}

\end{document}
