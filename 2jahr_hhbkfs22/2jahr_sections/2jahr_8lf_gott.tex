\section{Lernfeld 8 - Markt- und Kundenbeziehungen / GOTT} %Gottwald

\subsection{Grundbegriffe der Marktforschung}
Es wird zwischen Primär- (ield research) und Sekundärforschung (desk research) unterschieden. Bei der Primärforschung werden wie der englische Begriff nahelegt im Feld, das heißt vor Ort neue Daten erhoben. Im Gegensatz versucht die Sekundärforschung aus bereits bestehenden Daten vom Schreibtisch aus neue Information zu gewinnen.

% REWRITE!
\noindent Die Primärforschung zeichnet sich dadurch aus, dass sie im Gegensatz zur Sekundärforschung teuer und aufwändig ist. Daher wird sie in der Regel von großen Unternehmen oder Instituten durchgeführt. Zu den Methoden der Primärforschung gehören das Panel, die Beobachtung, die Befragung und das Experiment. Bei einem Panel wird versucht, Veränderungen über einen bestimmten Zeitraum zu messen und stützt sich dabei auf die Analyse von Beobachtungen und Experiment. Beobachtungen können teilnehmend oder nicht-teilnehmend sein. Teilnehmende Beobachtungen kommen einem Experiment nah. Es wird nochmals zwischen Feld- und Laborbeobachtung unterschieden. Eine nicht-teilnehmende Beobachtung ist das, was normalerweise unter einer Beobachtung verstanden wird. Sekundärforschung bedient sich internen oder externen Informationsquellen. Interne Quellen sind solche, die bereits im Unternehmen vorhanden sind. Externe Quellen sind Informationen, die von anderen Unternehmen oder Institutionen bezogen werden. Darunter fallen unter anderem Statistiken, aber auch Artikel aus der Fachpresse.

Marktanalyse (sekundär), Marktbeobachtung (primär), Marktprognose (sekundär)

\subsection{Instrumente der Marktforschung}

Bei den Instrumenten der Marktforschung kann zwischen Marktuntersuchung und Erhebungsverfahren unterschieden werden. Welche Instrumente dies im einzelnen sind, ist den beiden Tabellen zu entnehmen.

%Tabelle: Marktuntersuchung
\noindent \begin{tabular}{|p{\dimexpr 0.20\linewidth-2\tabcolsep}|
				 p{\dimexpr 0.25\linewidth-2\tabcolsep}|
				 p{\dimexpr 0.25\linewidth-2\tabcolsep}|
				 p{\dimexpr 0.30\linewidth-2\tabcolsep}|} %4 Spalten
	\hline
	\multicolumn{4}{|c|}{{{\bf{\large Marktuntersuchung}}}} \\
	\hline
	\multicolumn{2}{|c|}{Maßnahmen} & Erläuterung & Beispiele \\
	\hline
%Markterkundung
Markterkundung & betriebsintern & {\bf unsystemische} Untersuchung der eigenen Informationen und Daten der Abteilungen Einkauf, Produktion und Verkauf & Mitarbeiter bringen ihre Erkenntnisse anhand von Reise- und Marktberichten oder durch Stellungnahmen in Besprechungen ein. \\
	\hline
%Marktforschung
Marktforschung nach dem Umfang der Untersuchung & Marktbeobachtung
& systematische, {\bf zeitraumbezogene} Untersuchung & Ein Marktforschungsinstitut wird beauftragt, systematisch einen Monat lang die Reaktion des Marktes auf verschieden Produktinserate zu untersuchen. \\
	\cline{2-4}
%Marktanalyse
& Marktanalyse & systematische, {\bf zeitpunktbezogene} Untersuchung & Es soll systematisch an einem Tag eine Kundenbefragung durchgeführt und ausgewertet werden. \\
	\cline{2-4}
%Produktanalyse
& Produktanalyse & Systematische Untersuchung der Produkte hinsichtlich Preis-Leistungs-Verhältnis, Absatzchancen, Beratungsaufwand, Ertragschancen & Es soll herausgefunden werden, mit welchen Produkten man langfristig den Markt am besten bedienen und am meisten Gewinn erzielen kann. \\
%Konkurrenzforschung
	\cline{2-4}
& Konkurrenzforschung & Systematisch Untersuchung betrifft Produkte und -neuheiten sowie der Marktanteile und des Marktverhaltens der Mitbewerber. & Es soll systematisch untersucht werden, welche Mitbewerber einen Internetshop betreiben. \\
	\cline{2-4}
%Kundenanalyse
& Kundenanalyse, Bedarfs-, und Absatzforschung & Systematische Untersuchung des Marktes hinsichtlich des Gesamtbedarfs an Produkten auf dem Markt sowie zu Kaufmotiven und Absatzmöglichkeiten. & Es wird eine Untersuchung in Auftrag gegeben, ob Schüler ein Interesse an einem Computerführerschein haben und wie groß der Markt sein wird. \\
	\hline
%Marktprognose
Marktprognose & Trendberechnung & Aufgrund von vorhandenen Marktdaten wird versucht, einen Trend festzustellen und eine Prognose in die Zukunft abzugeben. & Bei Handys wurde in den letzten Jahren ein Marktwachstum von über 10\% festgestellt. Dank iPhone wird in den nächsten fünf Jahren mit einer Absatzsteigerung von 60\% gerechnet. \\
	\hline
\end{tabular}

%Tabelle: Erhebungsverfahren
\noindent \begin{tabular}{|p{\dimexpr 0.2\linewidth-2\tabcolsep}|
				 p{\dimexpr 0.20\linewidth-2\tabcolsep}|
				 p{\dimexpr 0.60\linewidth-2\tabcolsep}|} %3 Spalten
\hline
\multicolumn{3}{|c|}{{{\bf{\large Erhebungsverfahren / -arten}}}} \\
\hline
Umfang
& Vollerhebung & Für das zu untersuchende Merkmal wird die gesamte Grundmenge untersucht, z.B. werden alle Kunden befragt. \\
\cline{2-3}
& Teilerhebung & Es wird eine Stichprobe untersucht, wobei auf eine repräsentative Stichprobe Wert gelegt werden sollte. Selektion erfolgt durch Zufallsauswahl (z.B. jeder 100. Kunde) oder per Quote nach festgelegten Merkmalen (z.B. Alter: 40\% unter 20 Jahre). \\
\hline
Primärerhebung
& Befragung & schriftlich, mündlich, telefonisch, online anhand vorgegebener Fragen: Einfachauswahlfragen, Mehrfachauswahlfragen, Skalenfragen, Maßzahlen oder freie Fragen. \\
\cline{2-3}
& Interview & persönlich nach vorgegebnen Merkmalen oder Fragen; Vorteile: Erklärungen, Nachfragen möglich, individuell. \\
\cline{2-3}
& Beobachtung & Dauerbeobachtung oder mehrere Kurzbeobachtungen als Multimomentbeobachtung (kostengünstiger) zu einem bestimmten Beobachtungszweck. \\
\cline{2-3}
& Panel & Regelmäßige Befragung einer bestimmten Personengruppe erfolgt über einen längeren Zeitraum mit den gleichen Fragen. \\
\cline{2-3}
& Test/Experiment & Meinungserhebung aus einer (neutral gestellten) Testumgebung, z.B. werden Kunden neutral verpackte Produkte oder Warenproben zur Begutachtung vorgelegt oder zum Test oder zur Erprobung zur Verfügung gestellt. \\
\hline
Sekundärerhebung & Schon vorliegende Daten werden verwendet
& interne Daten, z.B. werden Kundendaten, Umsatzdaten, Messberichte, Reklamationen ausgewertet \\
\cline{3-3}
& & externe Daten, z.B. werden Daten der statistischen Ämter, der Kammer und Innungen, Verbände, Zeitschriften usw. ausgewertet \\
\hline
\end{tabular}
