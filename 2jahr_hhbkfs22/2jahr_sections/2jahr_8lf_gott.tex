\section{Lernfeld 8 - Markt- und Kundenbeziehungen / GOTT} %Gottwald

\subsection{Was ist Marketing?}

Der Begriff 'Marketing' umfasst alle Maßnahmen, die darauf ausgerichtet sind, den Absatz zu fördern respektive die absatzpolitischen Unternehmensziele zu erreichen. Die Aufgabe des Marketings besteht darin, das Unternehmen am Markt und am Kunden auszurichten. Es wird unterschieden zwischen strategischem und operativen Marketing. Beim \bf{strategischen Marketing} werden langfristige und grundlegende Entscheidungen getroffen:
\begin{tabular}{llr}
	1a & Welche Märkte und Kundensegmente sollen wir bedienen? & Marktforschung \\
	2a & Welche Produkte und Leistungen sollen angeboten werden? & Produkt- \& Sortimentspolitik \\
	3a & Wie soll sich das Unternehmen darstellen (Corporate Identity)? & Kommunikationspolitik \\
\end{tabular}

Im Gegensatz dazu werden beim \bf{operativen Marketing} kurzfristige Entscheidungen getroffen, um auf Veränderungen am Markt zu reagieren. In diesen Bereich fallen daher fragen der folgenden Art:
\begin{tabular}{llr}
	1b & Mit welchen Produkten können kurzfristige Marktchancen wahrgenommen werden? & Produkt- \& Sortimentspolitik \\
	2b & Welche Werbemittel sollen eingesetzt werden? & Kommunikationspolitik \\
	3b & Mit welchen Preisen soll auf Marktsituationen reagiert werden? & Preispolitik \\
	4b & Welcher Absatzweg soll situations- und kundengerecht gewählt werden? & Distributionspolitik \\
\end{tabular}

\subsection{Aufgabenbereiche des Marketings}

Zu den \bf{Aufgabenbereichen} des Marketings gehören unter anderem die \bf{Marktforschung}, \bf{Produkt- und Sortimentspolitik} sowie die \bf{Preispolitik}, die auch Kontrahierungspolitik genannt wird. Der erste der drei Aufgabenbereiche deckt die Frage 1a des strategischen Marketings ab. Die Fragen 2a und 1b, welche Produkte angeboten werden sollen und mit welchen Produkten sich kurzfristige Marktchancen realisieren lassen, wird im Rahmen des zweiten Aufgabenbereiches \ql Produkt- und Sortimentspolitik\qr, beantwortet. Wie sich ein Unternehmen darstellen will, d.h. welche Coporate Identity es nach außen verkörpern will, wird zusammen mit der Frage 2b im Bereich \bf{Kommunikationspolitik} behandelt. Im Aufgabenbereich drei, der Preispolitik wird Frage 3b beantwortet und im Rahmen der \bf{Distributionspolitik} wird entschieden, welche Absatzwege gewählt werden soll (Frage 4b).

\subsection{Zuständigkeiten für Marketing}

Im Unternehmen übernimmt die Geschäftsleitung oder diKey-Account-Managere Vertriebsleitung Aufgaben des Marketings und bezieht bei Bedarf Agenturen hinzu. Eine Marketingabteilung soll die Kundenbeziehungen verbessern und den Absatz fördern. Product-Manager sind für die Produktkonzeption sowie Markterschließung der Produkte und Leistunge verantwortlich. Key-Account-Manager sollen die Geschäftsbeziehungen zu ihren Schlüsselkunden pflegen und neue Kunden gewinnen.

\subsection{Marktforschung}
\subsubsection{Grundbegriffe der Marktforschung}
Es wird zwischen Primär- (field research) und Sekundärforschung (desk research) unterschieden. Bei der Primärforschung werden wie der englische Begriff nahelegt im Feld, das heißt vor Ort neue Daten erhoben. Im Gegensatz versucht die Sekundärforschung aus bereits bestehenden Daten vom Schreibtisch aus neue Information zu gewinnen.

\noindent Die Primärforschung zeichnet sich dadurch aus, dass sie im Gegensatz zur Sekundärforschung teuer und aufwändig ist. Daher wird sie in der Regel von großen Unternehmen oder Instituten durchgeführt. Zu den Methoden der Primärforschung gehören das Panel, die Beobachtung, die Befragung und das Experiment. Bei einem Panel wird versucht, Veränderungen über einen bestimmten Zeitraum zu messen und stützt sich dabei auf die Analyse von Beobachtungen und Experiment. Beobachtungen können teilnehmend oder nicht-teilnehmend sein. Teilnehmende Beobachtungen kommen einem Experiment nah. Es wird nochmals zwischen Feld- und Laborbeobachtung unterschieden. Eine nicht-teilnehmende Beobachtung ist das, was normalerweise unter einer Beobachtung verstanden wird. Sekundärforschung bedient sich internen oder externen Informationsquellen. Interne Quellen sind solche, die bereits im Unternehmen vorhanden sind. Externe Quellen sind Informationen, die von anderen Unternehmen oder Institutionen bezogen werden. Darunter fallen unter anderem Statistiken, aber auch Artikel aus der Fachpresse.

Im Rahmen der Primär- und Sekundärforschung gibt es zudem die Instrumente Marktanalyse (sekundär), Marktbeobachtung (primär) und Marktprognose (sekundär). Eine Marktanalyse hat das Ziel das Angebot und die Nachfrage sowie die Konkurrenz zu ermitteln. Die Marktbeobachtung bezieht sich wie ein Panel auf einen Zeitraum und gibt dabei Informationen über die Entwicklung eines Marktes. Bei einer Marktprognose wird versucht, Aussagen über die zukünftige Entwicklung eines Marktes zu treffen. Dabei wird auf Analysen und Beobachtungen zurückgegriffen und bekannte Regelmäßigkeiten extrapoliert.

\subsubsection{Instrumente der Marktforschung}
Bei den Instrumenten der Marktforschung kann zwischen Marktuntersuchung und Erhebungsverfahren unterschieden werden. Welche Instrumente dies im einzelnen sind, ist den beiden Tabellen zu entnehmen.

%Tabelle: Marktuntersuchung
\noindent \begin{tabular}{|p{\dimexpr 0.20\linewidth-2\tabcolsep}|
				 p{\dimexpr 0.25\linewidth-2\tabcolsep}|
				 p{\dimexpr 0.25\linewidth-2\tabcolsep}|
				 p{\dimexpr 0.30\linewidth-2\tabcolsep}|} %4 Spalten
	\hline
	\multicolumn{4}{|c|}{{{\bf{\large Marktuntersuchung}}}} \\
	\hline
	\multicolumn{2}{|c|}{Maßnahmen} & Erläuterung & Beispiele \\
	\hline
%Markterkundung
Markterkundung & betriebsintern & {\bf unsystemische} Untersuchung der eigenen Informationen und Daten der Abteilungen Einkauf, Produktion und Verkauf & Mitarbeiter bringen ihre Erkenntnisse anhand von Reise- und Marktberichten oder durch Stellungnahmen in Besprechungen ein. \\
	\hline
%Marktforschung
Marktforschung nach dem Umfang der Untersuchung & Marktbeobachtung
& systematische, {\bf zeitraumbezogene} Untersuchung & Ein Marktforschungsinstitut wird beauftragt, systematisch einen Monat lang die Reaktion des Marktes auf verschieden Produktinserate zu untersuchen. \\
	\cline{2-4}
%Marktanalyse
& Marktanalyse & systematische, {\bf zeitpunktbezogene} Untersuchung & Es soll systematisch an einem Tag eine Kundenbefragung durchgeführt und ausgewertet werden. \\
	\hline
Marktforschung nach dem Ziel der Untersuchung
%Produktanalyse
& Produktanalyse & Systematische Untersuchung der Produkte hinsichtlich Preis-Leistungs-Verhältnis, Absatzchancen, Beratungsaufwand, Ertragschancen & Es soll herausgefunden werden, mit welchen Produkten man langfristig den Markt am besten bedienen und am meisten Gewinn erzielen kann. \\
%Konkurrenzforschung
	\cline{2-4}
& Konkurrenzforschung & Systematisch Untersuchung betrifft Produkte und -neuheiten sowie der Marktanteile und des Marktverhaltens der Mitbewerber. & Es soll systematisch untersucht werden, welche Mitbewerber einen Internetshop betreiben. \\
	\cline{2-4}
%Kundenanalyse
& Kundenanalyse, Bedarfs-, und Absatzforschung & Systematische Untersuchung des Marktes hinsichtlich des {\bf Gesamtbedarfs an Produkten} auf dem Markt sowie zu Kaufmotiven und Absatzmöglichkeiten. & Es wird eine Untersuchung in Auftrag gegeben, ob Schüler ein Interesse an einem Computerführerschein haben und wie groß der Markt sein wird. \\
	\hline
%Marktprognose
Marktprognose & Trendberechnung & Aufgrund von vorhandenen Marktdaten wird versucht, einen Trend festzustellen und eine Prognose in die Zukunft abzugeben. & Bei Handys wurde in den letzten Jahren ein Marktwachstum von über 10\% festgestellt. Dank iPhone wird in den nächsten fünf Jahren mit einer Absatzsteigerung von 60\% gerechnet. \\
	\hline
\end{tabular}

%Tabelle: Erhebungsverfahren
\noindent \begin{tabular}{|p{\dimexpr 0.2\linewidth-2\tabcolsep}|
				 p{\dimexpr 0.20\linewidth-2\tabcolsep}|
				 p{\dimexpr 0.60\linewidth-2\tabcolsep}|} %3 Spalten
\hline
\multicolumn{3}{|c|}{{{\bf{\large Erhebungsverfahren / -arten}}}} \\
\hline
Umfang
& Vollerhebung & Für das zu untersuchende Merkmal wird die {\bf gesamte Grundmenge} untersucht, z.B. werden alle Kunden befragt. \\
\cline{2-3}
& Teilerhebung & Es wird eine Stichprobe untersucht, wobei auf eine {\bf repräsentative Stichprobe} Wert gelegt werden sollte. Selektion erfolgt durch Zufallsauswahl (z.B. jeder 100. Kunde) oder per Quote nach festgelegten Merkmalen (z.B. Alter: 40\% unter 20 Jahre). \\
\hline
Primärerhebung
& Befragung & schriftlich, mündlich, telefonisch, online anhand vorgegebener Fragen: Einfachauswahlfragen, Mehrfachauswahlfragen, Skalenfragen, Maßzahlen oder freie Fragen. \\
\cline{2-3}
& Interview & persönlich nach {\bf vorgegebnen Merkmalen} oder Fragen; Vorteile: Erklärungen, Nachfragen möglich, individuell. \\
\cline{2-3}
& Beobachtung & Dauerbeobachtung oder mehrere Kurzbeobachtungen als Multimomentbeobachtung (kostengünstiger) zu einem bestimmten Beobachtungszweck. \\
\cline{2-3}
& Panel & {\bf Regelmäßige Befragung} einer bestimmten Personengruppe erfolgt über einen längeren Zeitraum mit den gleichen Fragen. \\
\cline{2-3}
& Test/Experiment & Meinungserhebung aus einer (neutral gestellten) Testumgebung, z.B. werden Kunden neutral verpackte Produkte oder Warenproben zur Begutachtung vorgelegt oder zum Test oder zur Erprobung zur Verfügung gestellt. \\
\hline
Sekundärerhebung & Schon vorliegende Daten werden verwendet
& interne Daten, z.B. werden Kundendaten, Umsatzdaten, Messberichte, Reklamationen ausgewertet \\
\cline{3-3}
& & externe Daten, z.B. werden Daten der statistischen Ämter, der Kammer und Innungen, Verbände, Zeitschriften usw. ausgewertet \\
\hline
\end{tabular}


\subsection{Produkt- und Sortimentspolitik}

\subsubsection{Aspekte}

\begin{tabular}{lll}
	& Herstellung & Handel \\
	\hline
	Maßnahmen & {\bf Erw. d. Produktprog.} & {\bf Sortimentserw.} \\
	Zusammensetzung & ständig vs zeitweise & Kernsort. vs Randsort. \\
	{Struktur/\\Umfang} & \multicolumn{2}{l}{{\bf breit vs schmal und tief vs flach}} \\

\end{tabular}

\subsubsection{Maßnahmen}

\begin{itemize}
	\item {\bf Produktgestaltung}: (1) Qualität, (2) Aufmachung, (3) Verpackung, (4) Markierung
	\item {\bf begleitende Servicepolitik}: (1) Kundendienst, (2) Garantie, (3) Schulung
	\item {\bf prozessorientiert Produktpolitik}: (1) Innovation, (2) Differenzierung, (3) Diversifikation, (4) Variation, (5) Elimination
\end{itemize}

\subsubsection{Produktlebensphasen}

Es werden fünf Phasen im Leben eines Produktes unterschieden. Der Grundgedanke dabei ist, dass jedes Produkt, jede Produktgruppe und jede Produktklasse dieselben Phasen durchlaufen. Der Aussagewert ist entsprechend begrenzt, weil es sich schon im Ansatz um eine Idealisierung handelt. Daher dient es eher als idealtypische Veranschaulichung.

\begin{tabular}{llll}
	Phasen & Kapitalbedarf & Werbekosten & Rentabilität \\
	\hline
	Einführung & hoch & hoch & negativ \\
	Wachstum & mittel & hoch & positiv \\
	Reife & niedirg & mittel & sehr positiv \\
	Sättigung & gering & gering & sehr positiv \\
	Degeneration & gering & kaum & fallend \\
\end{tabular}

Der Übergang von der Einführungsphase in die Wachstumsphase findet am Point-of-Break-Even statt.


Jede Phase zeichnet sich durch eigenen Probleme und Stragien aus. 

\subsection{Kommunikationspolitik}

Werbung ist nur ein Aspekt der Kommunikationspolitik. Weitere Aspekte sind ...

\subsubsection{Werbung}

Werbung umfasst alle Maßnahmen, die die Bereitschaft verbessern, Produkte oder Leistungen zu kaufen, zu buchen oder zu bestellen.

Die Grundsätze der Werbung ist (1) Wirksamkeit, (2) Wahrheit, (3) Klarheit und (4) Wirtschaftlichkeit. Eine Werbemaßnahme sollte demnach also wirksam sein, sie sollte sachlich korrekt sein und nicht täuschend, klar und leicht verständlich sowie kostengünstig.

\paragraph{Werbeplanung}~\\

Mittel, Träger

Bei Werbung wird zwischen {\bf Werbemittel} und {\bf Werbeträger} unterschieden. Beispielsweise ist das Werbemittel \ql Werbespot\qr\ dasgleiche, wenn es über die Werbeträger \ql Fernsehn\qr\ oder \ql Kino\qr\ ausgestrahlt wird.

Es werden drei Werbearten unterschieden: (1) Einzelwerbung, (2) Direktwerbung und (3) Massenwerbung. 

Zu den vier {\bf Werbezielen} gehören die (1) Einführung, (2) Expansion, (3) Erhaltung und (4) das Image. Inhaltlich kann Werbung beispielsweise auf (1) den Preis, (2) eine Aktion, (3) ein Leitbild oder (4) ein bestimmtes Thema ausgerichtet sein ({\bf Werbeinhalt}).

\subsubsection{Verkaufsförderung / Sales Promotion}

Es wird zwischen (1) Verbraucher-, (2) Außendienst- und (3) Händler-Promotion unterschieden. Dabei werden beispielsweise den Verbrauchern Geschenke gemacht, den Außerndienstmitarbeitern Werbematerial an die Hand gegeben oder den Händlern Schulungen angeboten.

\subsubsection{Öffentlichkeitsarbeit / Public Relations}

Es gibt verschiedene Maßnahmen im Rahmen der Öffentlichkeitsarbeit.

Product Placement

Zudem fällt in diesen Bereich das {\bf Sponsoring}.

\subsubsection{Persönlicher Verkauf}


\subsection{Preis- und Konditionenpolitik}

Im Rahmen der {\bf Kontrahierungspolitik} (Preis- und Konditionenpolitik) finden alle Entscheidungen statt, die den Preis betreffen. Die Preisgestaltung kann auf verschiedene Weisen erfolgen.

\subsubsection{Preisgestaltung}

Bei der Preisgestaltung werden drei Faktoren unterschieden. (1) Kostenorientierung, (2) Nachfrageorientierung und (3) Konkurrenzorientierung. Bei der {\bf kostenorientierten Preisgestaltung} wird der Preis anhand der anfallenden Kosten ermittelt. Durch Zuschlagskalkulation wird der Angebots- bzw. Verkaufspreis ermittelt. Der Preis wird \qr\ from company to market\qr\ ermittelt. Handlungskosten sind alle Kosten, die bei einem Händler neben den Einstandspreisen der verkauften Waren regelmäßig anfallen. Durch Handlungskostenzuschlag sollen alle Kosten des Händlers gedeckt werden. Der Gewinnzuschlagssatz ist das Ergebnis von Erfahrungswerten. Die kostenorientierte Preisbildung ist mit einigen Problemen verbunden. Zum einen kann es sein, dass ein kalkulierter Preis sich als marktfern herausstellt und zum anderen könnte die Zahlungsbereitschaft des Kunden nicht vollständig ausgeschöpft werden, d.h. der Kunde wäre bereit gewesen mehr zu bezahlen. Daher ist die kostenorientierte Preisgestaltung eher dazu geeignet, das kostendeckende Minimum zu ermitteln. Auf einem Käufermarkt werden die Preise ausgehandelt. Ob die Kunden bereit sind einen bestimmten Preis zu bezahlen, lässt sich durch Marktforschung ermitteln. Dieses {\bf nachfrageorientierte Vorgehen} wird auch als \qr\ market to company\qr\ bezeichnet. Der Vorteil dabei besteht darin, dass so marktgerechte Preise ermittelt werden und dass Kunden eventuell bereit sind einen höheren Barverkaufspreis zu akzeptieren. Bei der {\bf konkurrenzorientierten Preisgestatlung} steht statt des Kundens die Konkurrenz im Mittelpunkt. Zur Marktbereinigung können die Verkaufspreise kurzzeitig unterhalb der Selbstkosten angesetzt werden. Solche Preise werden als '{\bf Dumpingpreise}' bezeichnet. Dieses Vorgehen ist sinnvoll, wenn anschließend eine positive wirtschaftliche Entwicklung erwartet wird.

\subsubsection{Preisbildung und -bündelung}

Wie tief Preise angesetzt werden können, wird durch die {\bf Preisuntergrenze} definiert. Die kurzfristige Preisuntergrenze liegt in Höhe der variablen Kosten. Der Deckungsbeitrag ist dann gleich Null. Die langfristige Preisuntergrenze liegt in Höhe der Selbstkosten. Diese Untergrenze kann dauerhaft genutzt werden, es wird dann aber kein Gewinn erzielt.

Eine {\bf Preisdifferenzierung} liegt vor, wenn ein Unternehmen für gleiche oder gleichartige Produkte unterschiedliche Preise verlangt, die sich nicht oder nicht gänzlich durch Qualitätsunterschiede begründen lassen. Die Preisdifferenzierung ist nur dann erfolgreich, wenn sich die Teilmärkte voneinander isolieren lassen. Im Rahmen der Preispolitik lassen sich Preise anhand von sechs Kategorien differenzieren:

\begin{enumerate}
	\item {\bf räumlich}: Veräußerung von Waren auf regional abgegrenzten Märkten zu verschieden hohen Preisen, z.B. Preisdifferenzierung  zwischen In- und Ausland.
	\item {\bf zeitlich}: Forderung verschieden hoher Preise für gleichartige Waren je nach der zeitlichen Nachfrage, z.B. Verleih von Kinofilmen.
	\item {\bf sachlich}: Preishöhe je nach dem Verwendungszweck der Produkte, z.B. verschiedene Strom- und Gastarife für Industrie- und Haushaltsverbrauch.
	\item {\bf persönlich}: Preisstellung je nach der marketingpolitischen Bedeutung (z.B. A- oder C-Kunden) und/oder den Absatzfunktionen der Zielgruppen, z.B. Groß- oder Einzelhandel.
	\item {\bf mengenmäßig}: Preisgestaltung je nach abgenommener Menge.
	\item {\bf verwendungsbezogen}: Je nach Verwendungszweck werden unterschiedliche Preise angesetzt. Beipiel: Strom-/Wasserpreise für Endverbraucher oder Industriekunden.
\end{enumerate}

Eine Preisstrategie beschreibt die langfristige Ausrichtung der Preisgestatlung. Es werden zwei Strategien je nach ihrer Dauer unterschieden, mit denen jeweils eine andere Stellung im Markt angestregt wird. Bei der {\bf Penetrationsstrategie} werden kurzfristig niedrige Preise angesetzt, um damit hohe Umsätze zu erwirtschaften. Dadurch wird der Marktbeitritt für Konkurrenten erschwert. Wird diese Strategie langfristig beibehalten, spricht man auch von einer {\bf Niedrigpreisstrategie}. Mit einer Niedrigpreisstragie wird die Kostenführerschaft im Markt angestrebt. Bei dieser Strategie sind eigene Innovationen kaum möglich, weshalb oft erfolgreiche Konkurrenzprodukte nachgeahmt werden. Im Gegensatz dazu steht die {\bf Skimming-Strategie}. Bei Produkteinführung werden sehr hohe Preise angesetzt und so hohe Gewinnspannen erzeugt. Dadurch wird die Kaufbereitschft der Early Adopter abgeschöpft. Wird Skimming dauerhaft eingesetzt, handelt es sich um eine {\bf Hochpreisstrategie}, mit der die Preisführerschaft angestrebt wird. Die Preisführerschaft ist nur bekannten Marken möglich. Damit einher geht die Möglichkeit innovative Produkte zu finanzieren.

Bei der {\bf Preisbündelung} werden verschieden Artikel zu einem Gesamtpreis angeboten. Ein Beispiel für gemischte Preisbündelung sind Briefpapier mit den passenden Umschlägen. Beides wäre einzeln zu höheren Kosten erhältlich.

\subsubsection{Preiselastizität}

Die Verkaufmenge wird durch den Verkaufspreis beeinflusst. Die Veränderung der Nachfrage in Abhängigkeit vom Verkaufspreis wird mithilfe der Preiselastizität dargestellt ({\bf Preiselastizität der Nachfrage}).\\
$Preiselastizität = (-1)\times(\frac{neue Menge - alte Menge}{alte Menge} / \frac{neuer Preis - alter Preis}{alter Preis})$\\
Die Preiselastizität wird mit $-1$ multipliziert, damit das Ergebnis positiv wird. Werte unter $1$ gelten als {\bf unelastisch} und Werte über $1$ entsprechend als {\bf elastisch}. Beispiele dafür sind Güter des Grundbedarfs ($<1$) und Güter des gehobenen Bedarfs wie beispielsweise Autos oder Unterhaltungselektronik ($>1$). Wenn die Elastizität $1$ beträgt, spricht man von {\bf isoelastischer Nachfrage}. Diese kommt in der Realität sehr selten vor. Sollte die Preiselastizität $0$ betragen, bedeutet dies eine {\bf vollkommen unelastische Nachfrage}. Ein Beispiel hierfür sind Medikamente. Nur weil ein Medikamet billiger wird, wird es nicht zwangsläufig häufiger genommen.

\subsection{Produktpolitik}

Eine Entscheidungsgrundlage im Rahmen der Produktpolitik bieten die {\bf Zielgruppenanalyse}, die {\bf Produktpositionierung}, die {\bf Portfolioanalyse}, die {\bf Produktlebenszyklus-Analyse} und die {\bf Produktprogrammstruktur-Analyse}.

\subsubsection{Zielgruppenanalyse}

In der Zielgruppenanalyse werden soziodemographische, geographische und psychographische Merkmale erfasst, um verschiedene Zielgruppen voneinander zu unterscheiden. Damit wird ermittelt, wer als Kunde für ein bestimmtes Produkt in Frage kommt.

\subsubsection{Produktpositionierung}

Die Produktpositionierung erfolgt anhand der unverwechselbaren Kennzeichnung relevanter Eigenschaften eines Produktes in den Köpfen der Verbraucher. Die Positionierung orientiert sich an den wahrgenommenen Leistungsmerkmalen. Zu den {\bf Kernelementen} der klassichen Produktpositionierung gehören die (1) {\bf Datenerhebung}, (2) {\bf Verdichtung der Daten} zu Grafiken, (3) die {\bf \textsc{Ist}-Analyse} der eigenen Position und der der Konkurrenz, (4) die Ermittelung der {\bf Idealposition} aus Kudnensicht und (5) die der {\bf Distanz zwischen \textsc{Ist} und Ideal.

{\bf Beispiel}: Neupositionierung einer Hotelkette.
\begin{enumerate}
	\item Marktsegmentierung anhand von Segmentierungskriterien. Segmentierungskriterien sind Kunden- sowie Bedürfnismerkmale. In diesem Fall verläuft die Segmentierung beispielsweise zwischen Privat- und Geschäftskunden , der vorhandenen Kaufkraft und dem Alter der Kundschaft.
	\item Ermittlung relevanter Kriterien für Hotelwahl. Anhand von Sekundärmaterials und Befragungen lässt sich das Nachfrageverhalten der anvisierten Kunden ermitteln. Für das Beispiel bedeutet dies, dass ermittelt werden kann, welche allgemeine Ausstattung im Hotel erwartet wird; bspw.: Tagungsräume, Konferenztechnik, Telekommunikationstechnik, Gastronomiemöglichkeiten, Freizeitmöglichkeiten
	\item Gewichtung der Kriterien (meist Vierfelder-Matrix)
	\item Bewertung der Konkurrenz 
	\item Verdichtung bzw. Reduzierung der Kriterien
	\item Einordnung des eigenen Hotels und der Konkurrenz bezüglich der ermittelten Dimensionen (Vierfelder-Matrix)
\end{enumerate}	 
	 
\subsubsection{Portfolioanalyse}

Mithilfe der Portfolioanalyse werden die eigenen Produkte im Vergleich zur Konkurrenz positioniert. Dabei werden strategische Geschäftseinheiten respektive die Produktlinien in einer Vier- oder Mehrfelder-Matrix verortet. Die Portfolioanalyse wird auch Geschäftsfeld-Analyse genannt.

\paragraph{Boston-Consulting-Group: Marktwachstum-Marktanteil-Portfolio}~\\

Die Boston-Consulting-Group hat das \ql Marktwachstum-Marktanteil-Portfolio\qr\ entwickelt. Dabei wird das Marktwachstum mit dem relativen Marktanteil eines Produktes verglichen. Der Marktanteil und das Marktwachstum werden je nach niedrig und hoch unterteilt. Dadurch entstehen wird Felder:

\begin{tabular}{llll}
Marktanteil & Marktwachstum & & Beschreibung \\
hoch & hoch & Stars & gute Wachstumschancen; beanspruchen viele Ressourcen und erwirtschaften kaum Überschüsse \\
hoch & niedrig & Cash Cows & erfolgreiche und etablierte Produkte; sichern kurzfristig den Erfolg \\
niedrig & hoch & Question Marks & binden finanzielle Mitteln; Zukunft ungewiss \\
niedrig & niedrig & Poor Dogs & & \\
\end{tabular}

Innerhalb der vier Felder werden die Produkte als Kreis dargestellt. Die Größe des Kreises entspricht dabei der Bedeutung des Produktes für das Unternehmen. McKinsey kritisiert an der Analyse die Eindimensionalität.
	
\paragraph{McKinsey: Marktattraktivität gegenüber relativem Wettbewerbsvorteil}

McKinsey hat als Alternative zum Marktwachstum-Marktanteil-Portfolio die Analyse der Marktattraktivität gegenüber dem relativen Wettbewerbsvorteil entwickelt. Die {\bf Marktattraktivität} wird bestimmt durch (1) das Marktwachstum, (2) die Marktgröße, (3) die Marktqualität und (4) die Umweltsituation. Die {\bf realtiven Wettbewerbsvorteile} ergeben sich aus der Größe (1) der relativen Marktposition, (2) des relativen Produktpotenzials und (3) der relativen Produktqualität. Die beiden Faktoren werden je nach niedrig, mittel und hoch unterschieden.

\subsubsection{Produktlebenszyklus-Analyse}

Die Grundidee der Produktlebenszyklus-Analyse besteht darin, dass Produkte, Produktgruppen und Produktklassen dieselben Phasen durchlaufen. Durch die Relation von Umsatz zu Gewinn werden fünf Phasen unterschieden.

\begin{enumerate}
	\item Einführungsphase: Beginnt mit Release, endet mit erreichen der Gewinnschwelle (Entwicklung und Marketing müssen sich erst amortisieren). Wenn neuartiges Produkt marktbeherrschend, dann Preiselastizität gering (iPhone). Kunden: Aufgeschlossen, innovationswillig - Innovatoren.
	\item Wachstumsphase: Bekanntheit nimmt zu, Produkt erwirtschaftet Gewinn me-too-Produkte erscheinen, Preiselastizität nimmt zu Bekanntheit/Festigung des Images: mehr Kunden. Kunden: erste Stammkunden - Early Adopter
	\item Reifephase: Wachstumsrate und Gewinn gehen zurück, Umsatz immer noch steigend. Preiselastizität der Nachfrage nimmt stark zu,  Preispolitik wird effektives absatzpolitisches Instrument. Steigende Anzahl der Konkurrenten. Produktdifferenzierung kann Umsatz weiter erhöhen. Kunden: konservative Einstellung - frühe Mehrheit
	\item Sättigungsphase: Beginnt, wenn Umsatz nicht mehr wächst, Nachfrage stagniert, Preiselastizität der Nachfrage ist hier am größten. Kunden: späte Mehrheit
	\item Degenerationsphase: Bedürfnis wird durch andere Produkte befriedigt Nachfrage sinkt rapide, Umsatz sinkt, Verluste Preiselastizität nähert sich dem Nullpunkt. Kunden: Nachzügler 
\end{enumerate}

Der Aussagewert der Produktlebenszyklus-Analyse ist stark begrenzt, da es sich um eine Idealisierung handelt. Gegenbeispiele sind unter anderem Flops und Relaunches. Flops zeichnen sich durch ein schnelles Wachstum und einen schnellen Rückgang aus.

\subsubsection{Produktprogrammstruktur-Analyse}

In der Produktprogrammstruktur-Analyse werden die Altersstruktur, die Umsatzstruktur, die Kundenstruktur, die Deckungsbeitragsstruktur und die Struktur der Geschäftsflelder verglichen. Die Alterstruktur vergleicht dabei die Produktlebenszyklen miteinander. Die Umsatzstruktur wird als Lorenz-Kurve dargestellt und zeigt die Verteilung des Umsatzes auf die verschiedenen Produkte. Die Kundenstruktur wird auch ABC-Analyse genannt, weil die Kunden anhand von Umsatz, Gewinn und strategischer Bedeutung in drei Gruppen aufgeteilt werden. Der Deckungsbeitrag zeigt den Anteil der Produkte am Erfolg des Unternehmens.

\subsubsection{Produktinnovation}

Produktinnovationen sind teuer und ressourcenintensiv. Daher ist sorgfältige Planung notwendig. Gründe, die Produktinnovationen notwendig machen, sind beispielsweise \dots

\begin{itemize}
	\item Wachstumsstrategie
	\item Produkt ist veraltet
	\item Patente laufen aus
	\item Produkt-Mix ist einseitig
	\item technische Neuerungen
	\item geänderte rechtliche Restriktionen
	\item veränderte Kundenansprüche
	\item unternehmensinterne Restriktionen
	\item Zufallsentwicklungen
\end{itemize}

Im Rahmen der Produktinnovation wird zwischen Produktdifferenzierung und Produktdiversifizierung unterschieden. {\bf Produktdifferenzierung} bedeutet, dass zu den vorhandenen Produkten zusätzliche Produkte entwickelt und vermarktet werden (Produktvariationen). Dieses Vorgehen wird auch als {\bf Marktsegmentierung} bezeichnet. Marktsegmentierung bezeichnet die Aufteilung eines Gesamtmarktes in hinsichtlich ihrer Marktreaktionen intern homogenener und untereinander heterogener Untergruppen. Gründe dafür können sein:

\begin{itemize}
	\item Eroberng neuer Märkte mit bekannten Produkten
	\item konsequente Marktsegmentierung
	\item Ausnutzung von Synergieeffekten
	\item technischer Fortschritt
	\item Anpassung an Mode
	\item {rechtliche Unterschiede in Ländern}
	\item Einführung eines erfolgreichen Nischenproduktes
\end{itemize}

{\bf Produktdiversifizierung} bezeichnet im Gegensatz die Entwicklung und Vermarktung neuer Produkte. Innerhalb der Produktdiversifikation wird zwischen {\bf horizontaler}, {\bf vertikaler} und {\bf lateraler} Diversifikation unterschieden. Dabei bezeichnet das erstere die Entwicklung von Produkten derselben Wirtschaftsstufe, das zweite die von Produkten höherer oder niedrigerer Stufe und letzteres die Entwicklung von sachlich unzusammenhängenden Produkten.
			 
			 
\subsubsection{Verkaufspreiskalkulation}

Im Rahmen der Verkaufspreiskalkulation werden einige Begriffe unterschieden.

\begin{itemize}
	\item Handlungskosten sind alle Kosten, die bei einem Großhändler zu Erbringung seiner Leistung neben den Einstandspreisen der verkauften Waren regelmäßig anfallen.
	\item Der Handlungskostenzuschlag wird auf Basis der Kostensituation in der Vergangenheit mit einem Betriebsabrechnungsbogen ermittelt.
	\item Der Gewinnzuschlagssatz ist das Ergebnis einer Entscheidung auf Basis von Erfahrungswerten.
	\item Deckungsbeitrag
	\item Selbstkosten
	\item 
	\item 
	\item 
	\item 
	\item 
	\item 
	\item 
\end{itemize}

\paragraph{Vorwärtskalkulation}~\\
\paragraph{Vorwärtskalkulation mit Rabatttafel}~\\
\paragraph{Rückwärtskalkulation}~\\
\paragraph{Formelsammlung}~\\		 
			 
			 
\subsection{Distributionspolitik}


\subsubsection{eCommerce}

Im eCommerce wird zwischen drei Akteuren unterschieden. Erstens dem Konsumenten ({\bf Consumer C}), zweitens den Unternehmen ({\bf Business B}) und drittens den öffentlichen Behörden ({\bf Administration A}). Daraus ergeben sich neun Beziehungen. Diese werden beispielsweise als {\bf B2A} bezeichnet. Das bedeutet in diesem Fall, dass eine öffentliche Behörde einem privatwirtschaftlichen Unternehmen eine Dienstleistung anbietet. Im Allgemeinen bezeichnet die Darstellung {\bf x2y}, dass x etwas bei y nachfragt oder y anbietet. Ein weiteres Beispiel wäre {\bf C2A}: Verbraucher fragen Informationen bei einer Behörde an.
