\section{Lernfeld 8 - Markt- und Kundenbeziehungen / GOTT} %Gottwald

\subsection{Grundbegriffe der Marktforschung}
Es wird zwischen Primär- (ield research) und Sekundärforschung (desk research) unterschieden. Bei der Primärforschung werden wie der englische Begriff nahelegt im Feld, das heißt vor Ort neue Daten erhoben. Im Gegensatz versucht die Sekundärforschung aus bereits bestehenden Daten vom Schreibtisch aus neue Information zu gewinnen.

\noindent Die Primärforschung zeichnet sich dadurch aus, dass sie im Gegensatz zur Sekundärforschung teuer und aufwändig ist. Daher wird sie in der Regel von großen Unternehmen oder Instituten durchgeführt. Zu den Methoden der Primärforschung gehören das Panel, die Beobachtung, die Befragung und das Experiment. Bei einem Panel wird versucht, Veränderungen über einen bestimmten Zeitraum zu messen und stützt sich dabei auf die Analyse von Beobachtungen und Experiment. Beobachtungen können teilnehmend oder nicht-teilnehmend sein. Teilnehmende Beobachtungen kommen einem Experiment nah. Es wird nochmals zwischen Feld- und Laborbeobachtung unterschieden. Eine nicht-teilnehmende Beobachtung ist das, was normalerweise unter einer Beobachtung verstanden wird.