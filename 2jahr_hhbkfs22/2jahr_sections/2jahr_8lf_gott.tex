\section{Lernfeld 8 - Markt- und Kundenbeziehungen / GOTT} %Gottwald

\subsection{Grundbegriffe der Marktforschung}
Es wird zwischen Primär- (ield research) und Sekundärforschung (desk research) unterschieden. Bei der Primärforschung werden wie der englische Begriff nahelegt im Feld, das heißt vor Ort neue Daten erhoben. Im Gegensatz versucht die Sekundärforschung aus bereits bestehenden Daten vom Schreibtisch aus neue Information zu gewinnen.

\noindent Die Primärforschung zeichnet sich dadurch aus, dass sie im Gegensatz zur Sekundärforschung teuer und aufwändig ist. Daher wird sie in der Regel von großen Unternehmen oder Instituten durchgeführt. Zu den Methoden der Primärforschung gehören das Panel, die Beobachtung, die Befragung und das Experiment. Bei einem Panel wird versucht, Veränderungen über einen bestimmten Zeitraum zu messen und stützt sich dabei auf die Analyse von Beobachtungen und Experiment. Beobachtungen können teilnehmend oder nicht-teilnehmend sein. Teilnehmende Beobachtungen kommen einem Experiment nah. Es wird nochmals zwischen Feld- und Laborbeobachtung unterschieden. Eine nicht-teilnehmende Beobachtung ist das, was normalerweise unter einer Beobachtung verstanden wird. Sekundärforschung bedient sich internen oder externen Informationsquellen. Interne Quellen sind solche, die bereits im Unternehmen vorhanden sind. Externe Quellen sind Informationen, die von anderen Unternehmen oder Institutionen bezogen werden. Darunter fallen unter anderem Statistiken, aber auch Artikel aus der Fachpresse.

Im Rahmen der Primär- und Sekundärforschung gibt es zudem die Instrumente Marktanalyse (sekundär), Marktbeobachtung (primär) und Marktprognose (sekundär). Eine Marktanalyse hat das Ziel das Angebot und die Nachfrage sowie die Konkurrenz zu ermitteln. Die Marktbeobachtung bezieht sich wie ein Panel auf einen Zeitraum und gibt dabei Informationen über die Entwicklung eines Marktes. Bei einer Marktprognose wird versucht, Aussagen über die zukünftige Entwicklung eines Marktes zu treffen. Dabei wird auf Analysen und Beobachtungen zurückgegriffen und bekannte Regelmäßigkeiten extrapoliert.

\subsection{Instrumente der Marktforschung}

Bei den Instrumenten der Marktforschung kann zwischen Marktuntersuchung und Erhebungsverfahren unterschieden werden. Welche Instrumente dies im einzelnen sind, ist den beiden Tabellen zu entnehmen.

%Tabelle: Marktuntersuchung
\noindent \begin{tabular}{|p{\dimexpr 0.20\linewidth-2\tabcolsep}|
				 p{\dimexpr 0.25\linewidth-2\tabcolsep}|
				 p{\dimexpr 0.25\linewidth-2\tabcolsep}|
				 p{\dimexpr 0.30\linewidth-2\tabcolsep}|} %4 Spalten
	\hline
	\multicolumn{4}{|c|}{{{\bf{\large Marktuntersuchung}}}} \\
	\hline
	\multicolumn{2}{|c|}{Maßnahmen} & Erläuterung & Beispiele \\
	\hline
%Markterkundung
Markterkundung & betriebsintern & {\bf unsystemische} Untersuchung der eigenen Informationen und Daten der Abteilungen Einkauf, Produktion und Verkauf & Mitarbeiter bringen ihre Erkenntnisse anhand von Reise- und Marktberichten oder durch Stellungnahmen in Besprechungen ein. \\
	\hline
%Marktforschung
Marktforschung nach dem Umfang der Untersuchung & Marktbeobachtung
& systematische, {\bf zeitraumbezogene} Untersuchung & Ein Marktforschungsinstitut wird beauftragt, systematisch einen Monat lang die Reaktion des Marktes auf verschieden Produktinserate zu untersuchen. \\
	\cline{2-4}
%Marktanalyse
& Marktanalyse & systematische, {\bf zeitpunktbezogene} Untersuchung & Es soll systematisch an einem Tag eine Kundenbefragung durchgeführt und ausgewertet werden. \\
	\hline
Marktforschung nach dem Ziel der Untersuchung
%Produktanalyse
& Produktanalyse & Systematische Untersuchung der Produkte hinsichtlich Preis-Leistungs-Verhältnis, Absatzchancen, Beratungsaufwand, Ertragschancen & Es soll herausgefunden werden, mit welchen Produkten man langfristig den Markt am besten bedienen und am meisten Gewinn erzielen kann. \\
%Konkurrenzforschung
	\cline{2-4}
& Konkurrenzforschung & Systematisch Untersuchung betrifft Produkte und -neuheiten sowie der Marktanteile und des Marktverhaltens der Mitbewerber. & Es soll systematisch untersucht werden, welche Mitbewerber einen Internetshop betreiben. \\
	\cline{2-4}
%Kundenanalyse
& Kundenanalyse, Bedarfs-, und Absatzforschung & Systematische Untersuchung des Marktes hinsichtlich des Gesamtbedarfs an Produkten auf dem Markt sowie zu Kaufmotiven und Absatzmöglichkeiten. & Es wird eine Untersuchung in Auftrag gegeben, ob Schüler ein Interesse an einem Computerführerschein haben und wie groß der Markt sein wird. \\
	\hline
%Marktprognose
Marktprognose & Trendberechnung & Aufgrund von vorhandenen Marktdaten wird versucht, einen Trend festzustellen und eine Prognose in die Zukunft abzugeben. & Bei Handys wurde in den letzten Jahren ein Marktwachstum von über 10\% festgestellt. Dank iPhone wird in den nächsten fünf Jahren mit einer Absatzsteigerung von 60\% gerechnet. \\
	\hline
\end{tabular}

%Tabelle: Erhebungsverfahren
\noindent \begin{tabular}{|p{\dimexpr 0.2\linewidth-2\tabcolsep}|
				 p{\dimexpr 0.20\linewidth-2\tabcolsep}|
				 p{\dimexpr 0.60\linewidth-2\tabcolsep}|} %3 Spalten
\hline
\multicolumn{3}{|c|}{{{\bf{\large Erhebungsverfahren / -arten}}}} \\
\hline
Umfang
& Vollerhebung & Für das zu untersuchende Merkmal wird die gesamte Grundmenge untersucht, z.B. werden alle Kunden befragt. \\
\cline{2-3}
& Teilerhebung & Es wird eine Stichprobe untersucht, wobei auf eine repräsentative Stichprobe Wert gelegt werden sollte. Selektion erfolgt durch Zufallsauswahl (z.B. jeder 100. Kunde) oder per Quote nach festgelegten Merkmalen (z.B. Alter: 40\% unter 20 Jahre). \\
\hline
Primärerhebung
& Befragung & schriftlich, mündlich, telefonisch, online anhand vorgegebener Fragen: Einfachauswahlfragen, Mehrfachauswahlfragen, Skalenfragen, Maßzahlen oder freie Fragen. \\
\cline{2-3}
& Interview & persönlich nach vorgegebnen Merkmalen oder Fragen; Vorteile: Erklärungen, Nachfragen möglich, individuell. \\
\cline{2-3}
& Beobachtung & Dauerbeobachtung oder mehrere Kurzbeobachtungen als Multimomentbeobachtung (kostengünstiger) zu einem bestimmten Beobachtungszweck. \\
\cline{2-3}
& Panel & Regelmäßige Befragung einer bestimmten Personengruppe erfolgt über einen längeren Zeitraum mit den gleichen Fragen. \\
\cline{2-3}
& Test/Experiment & Meinungserhebung aus einer (neutral gestellten) Testumgebung, z.B. werden Kunden neutral verpackte Produkte oder Warenproben zur Begutachtung vorgelegt oder zum Test oder zur Erprobung zur Verfügung gestellt. \\
\hline
Sekundärerhebung & Schon vorliegende Daten werden verwendet
& interne Daten, z.B. werden Kundendaten, Umsatzdaten, Messberichte, Reklamationen ausgewertet \\
\cline{3-3}
& & externe Daten, z.B. werden Daten der statistischen Ämter, der Kammer und Innungen, Verbände, Zeitschriften usw. ausgewertet \\
\hline
\end{tabular}

\subsection{Vollkostenrechnung}
\subsubsection{Vorwärtskalkulation}
\subsubsection{Rückwärtskalkulation}


\subsection{Produktpolitik: Preisbildung und Preisstrategien}

Was ist Marketing?

Im Rahmen der Kontrahierungspolitik (Preis- und Konditionenpolitik) finden alle Entscheidungen statt, die den Preis betreffen. Die Preisgestaltung kann auf verschiedene Weisen erfolgen.

\subsubsection{Preisgestaltung}

Bei der Preisgestaltung werden drei Faktoren unterschieden. 1) Kostenorientierung, 2) Nachfrageorientierung und 3) Konkurrenzorientierung. Bei dem ersten Faktor wird der Preis anhand der anfallenden Kosten ermittelt. Durch Zuschlagskalkulation wird der Angebots- bzw. Verkaufspreis ermittelt. Der Preis wird \qr\ from company to market\qr\ ermittelt. Handlungskosten sind alle Kosten, die bei einem Händler neben den Einstandspreisen der verkauften Waren regelmäßig anfallen. Durch Handlungskostenzuschlag sollen alle Kosten des Händlers gedeckt werden. Der Gewinnzuschlagssatz ist das Ergebnis von Erfahrungswerten. Die kostenorientierte Preisbildung ist mit einigen Problemen verbunden. Zum einen kann es sein, dass ein kalkulierter Preis sich als marktfern herausstellt und zum anderen könnte die Zahlungsbereitschaft des Kunden nicht vollständig ausgeschöpft werden, d.h. der Kunde wäre bereit gewesen mehr zu bezahlen.

\begin{itemize}
	\item Preisführer
	\item Preisfolger
	\item Preiskämpfer
\end{itemize}



\begin{itemize}
	\item Skimming (Abschöpfung):
	\item Penetration (Marktdurchdringung):
\end{itemize}

\subsubsection{Preisdifferenzierung}

Eine Preisdifferenzierung liegt vor, wenn ein Unternehmen für gleiche oder gleichartige Produkte unterschiedliche Preise verlangt, die sich nicht oder nicht gänzlich durch Qualitätsunterschiede begründen lassen. Im Rahmen der Preispolitik lassen sich Preise anhand von vier Kategorien differenzieren
\begin{enumerate}
	\item {\bf räumlich}: Veräußerung von Waren auf regional abgegrenzten Märkten zu verschieden hohen Preisen, z.B. Preisdifferenzierung  zwischen In- und Ausland.
	\item {\bf zeitlich}: Forderung verschieden hoher Preise für gleichartige Waren je nach der zeitlichen Nachfrage, z.B. Verleih von Kinofilmen.
	\item {\bf sachlich}: Preishöhe je nach dem Verwendungszweck der Produkte, z.B. verschiedene Strom- und Gastarife für Industrie- und Haushaltsverbrauch.
	\item {\bf persönlich}: Preisstellung je nach der marketingpolitischen Bedeutung (z.B. A- oder C-Kunden) und/oder den Absatzfunktionen der Zielgruppen, z.B. Groß- oder Einzelhandel.
\end{enumerate}

\subsubsection{Preiselastizität}

$Preiselastizität = (-1)\times(\frac{neue Menge - alte Menge}{alte Menge} / \frac{neuer Preis - alter Preis}{alter Preis})$\\
Die Preiselastizität wird mit $-1$ multipliziert, damit das Ergebnis positiv wird. Werte unter $1$ gelten als unelastisch und Werte über $1$ entsprechend als elastisch. Beispiele dafür sind Güter des Grundbedarfs ($<1$) und Güter des gehobenen Bedarfs wie beispielsweise Autos oder Unterhaltungselektronik ($>1$). Wenn die Elastizität $1$ beträgt, spricht man von isoelastischer Nachfrage. Diese kommt in der Realität sehr selten vor. Sollte die Preiselastizität $0$ betragen, bedeutet dies eine vollkommen unelastische Nachfrage. Ein Beispiel hierfür sind Medikamente. Nur weil ein Medikamet billiger wird, wird es nicht häufiger genommen.

\subsubsection{Analysen}

\begin{itemize}
	\item Zielgruppenanalyse
	\item Portfolioanalyse
	\item Produktlebenszyklus-Analyse
	\item Produktprogrammstruktur-Analyse
	\item Geschäftsfeld-Analyse
\end{itemize}



\subsubsection{Zielgruppenanalyse}
Wer: Soziodemographische und psychographische Merkmale      

Produktpositionierung
	Unverwechselbare Kennzeichnung (relevante Eigenschaften) eines
	Produktes in den Köpfen der Verbraucher. Positionierung orientiert
	sich an den wahrgenommenen Leistungsmerkmalen.
 Kernelemente der klassische Positionierung
	- 1. Datenerhebung: Welche Eigenschaften nimmt der Kunde wahr?
	- 2. Verdichtung der Daten zu Grafiken
	- 3. Platzierung der eigenen Produkte sowie die der Konkurrenz
	- 4. Idealposition aus Kundensicht ermitteln
	- 5. Distanz zwischen Ist und Ideal ermitteln
 Beispiel: Neupositionierung einer Hotelkette
	- 1. Marktsegmentierung anhand von Segmentierungskriterien
		Segmentierungskriterien: Kunden-/Bedürfnismerkmale
		# sozio-, demo-, psychographische Merkmale
		- Firmen- vs Privatkunden
		- Einkommen: exklusiv vs normal
		- Alter: Junge vs ältere Kundschaft
	- 2. Ermittlung relevanter Kriterien für Hotelwahl
		# Sekundärmaterial, Befragungen
		- allg. Ausstattung
		- Tagungsräume
		- Konferenztechnik
		- Telekommunikationstechnik
		- Grastronomiemöglichkeiten
		- Freizeitmöglichkeiten
	- 3. Gewichtung der Kriterien (meist Vierfelder-Matrix)
	- 4. Bewertung der Konkurrenz
	- 5. Verdichtung der Kriterien (Faktorenanalyse)
	- 6. Einordnung des eigenen Hotels und der Konkurrenz bezüglich der
	 ermittelten Dimensionen (Vierfelder-Matrix)
Portfolioanalyse
	strategische Geschäftseigenheiten bzw. die Produktlinie wird in einer
	Vierfelder-Matrix verortet
	- Boston-Consulting-Group: Marktwachstum-Marktanteil-Portfolio
		- vergleicht Marktwachstum mit relativem Marktanteil
		* jeweils hoch und niedrig
					Marktanteil	/ Marktwachstum
			* Stars			hoch	/ hoch
			 - beanspruchen viele Ressourcen und erwirtschaften
			   kaum Überschüsse
			* Cash Cows		hoch	/ niedrig
			 - erfolgreiche und etablierte Produkte
			 - sichern kurzfristig den Erfolg des Unternehmens
			* Question Marks	niedrig	/ hoch
			 - binden finanzielle Mittel, Zukunft ungewiss
			* Poor Dogs		niedrig	/ niedrig

		- Kreisposition entspr Verortung, Größe entspr Bedeutung
	- McKinsey kritisiert Eindimensionalität
		- stattdessen Marktattraktivität vs rel. Wettbewerbsvorteil
			- Marktattraktivität: bestimmt durch
			 * Marktwachstum
			 * Marktgröße
			 * Marktqualität
			 * Umweltsituation
			- relative Wettbewerbsvorteile
			 * Größe der relativen Marktposition
			 * relatives Produktionspotenzial
			 * relative Personalqualität
		- je niedrig, mittel, hoch
Produktlebenszyklus-Analyse
	- Grundidee: Produkte, P.Gruppen und P.Klassen durchlaufen dieselben
	 Phasen; Umsatz rel zu Gewinn -> 5 Phasen
		- Einführungsphase
			 Beginnt mit Release, endet mit erreichen der
			 Gewinnschwelle (Entwicklung und Marketing müssen sich
			 erst amortisieren). Wenn neuartiges Produkt markt-
			 beherrschend, dann Preiselastizität gering (iPhone)
				Kunden: Aufgeschlossen, innovationswillig
				- Innovatoren
		- Wachstumsphase
			 Bekanntheit nimmt zu, Produkt erwirtschaftet Gewinn
			 me-too-Produkte erscheinen, Preiselastizität nimmt zu
			 Bekanntheit/Festigung des Images: mehr Kunden
				Kunden: erste Stammkunden, Early Adopter
		- Reifephase
			 Wachstumsrate und Gewinn gehen zurück
			 * Umsatz immer noch steigend
			 Preiselastizität der Nachfrage nimmt stark zu,
			 Preispolitik wird effektives absatzpolitisches Instr
			 Steigende Anzahl der Konkurrenten, Differenzierung
				Kunden: konservative Einstellung,
					frühe Mehrheit
		- Sättigungsphase
			 Beginnt, wenn Umsatz nicht mehr wächst, Nachfrage
			 stagniert, Preiselastizität der Nachfrage ist hier am
			 größten
				Kunden: späte Mehrheit
		- Degenerationsphase
			 Bedürfnis wird durch andere Produkte befriedigt
			 Nachfrage sinkt rapide, Umsatz sinkt, Verluste
			 Preiselastizität nähert sich dem Nullpunkt
				Kunden: Nachzügler 
	- Aussagewert begrenzt, da Idealisierung
		- Flop: schnelles Wachstum, schneller Rückgang
		- Relaunch

\subsubsection{Produktlebensphasen}

Es werden fünf Phasen im Leben eines Produktes unterschieden. Der Grundgedanke dabei ist, dass jedes Produkt, jede Produktgruppe und jede Produktklasse dieselben Phasen durchlaufen. Der Aussagewert ist entsprechend begrenzt, weil es sich schon im Ansatz um eine Idealisierung handelt. Daher dient es eher als idealtypische Veranschaulichung.

\begin{enumerate}
	\item Einführungsphase
	\item Wachstumsphase
	\item Reifephase
	\item Sättigungsphase
	\item Degenerationsphase
\end{enumerate}

Produktprogrammstruktur-Analyse
	- vergleich Altersstruktur, Umsatzstruktur, Kundenstruktur,
	 Deckungsbeitragsstruktur und Struktur der Geschäftsfelder
		- Altersstruktur: vergleicht Produktlebenszyklus
		- Umsatzstruktur: Umsatzprofile - welches Produkt
		 erwirtschaftet wie viel Prozent des Gewinns (Lorenz-Kurve)
		- Kundenstruktur: ABC-Analyse und Einteilung der Kunden
		 Einteilung: Umsatz, Gewinn, stragetische Bedeutung
		- Deckungsbeitrag: Anteil des Produktes am Erfolg des
		 Unternehmens (Deckungsbeitragsrechnung)
Geschäftsfeld-Analyse
	- wird mithilfe der Portfolio-Analyse durchgeführt 

\subsubsection{Produktinnovation}

Teuer und ressourcenintensiv, daher sorgfältige Planung
Gründe:
	- Wachstumsstrategie
	- Produkt ist veraltet
	- Patente laufen aus
	- Produkt-Mix ist einseitig
	- technische Neuerungen
	- geänderte rechtliche Restriktionen
	- veränderte Kundenansprüche
	- unternehmensinterne Restriktionen
	- Zufallsentwicklungen
Unterscheidung:
	- Produktdifferenzierung
		- Entwicklung und Vermarktung zusätzlich zu den vorhandenen
		 Produkten (Marktsegmentierung)
		- Gründe:
			- Neue Märkte mit bekannten Produkten
			- konsequente Marktsegmentierung
			- Ausnutzung von Synergieeffekten
			- technischer Fortschritt
			- Anpassung an Mode
			- rechtliche Unterschiede in Ländern
			- Einführung eines erfolgreichen Nischenproduktes
	- Produktdiversifizierung
		- Entwicklung und Vermarktung neuer Produkte
		- horizontale Diversifikation
			- Produkte derselben Wirtschaftsstufe
		- vertikale Diversifikation
			- Produkte höherer oder niedrigerer Stufe
		- laterale Diversifikation
			- zwischen den Produkten besteht kein sachlicher
			 Zusammenhang

			 
			 
\section{Distributionspolitik}
