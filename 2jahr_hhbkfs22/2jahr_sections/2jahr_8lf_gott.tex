\section{Lernfeld 8 - Markt- und Kundenbeziehungen / GOTT} %Gottwald

\subsection{Was ist Marketing?}

Der Begriff 'Marketing' umfasst alle Maßnahmen, die darauf ausgerichtet sind, den Absatz zu fördern respektive die absatzpolitischen Unternehmensziele zu erreichen. Die Aufgabe des Marketings besteht darin, das Unternehmen am Markt und am Kunden auszurichten. Es wird unterschieden zwischen strategischem und operativen Marketing. Beim \bf{strategischen Marketing} werden langfristige und grundlegende Entscheidungen getroffen:
\begin{tabular}{llr}
	1a & Welche Märkte und Kundensegmente sollen wir bedienen? & Marktforschung \\
	2a & Welche Produkte und Leistungen sollen angeboten werden? & Produkt- \& Sortimentspolitik \\
	3a & Wie soll sich das Unternehmen darstellen (Corporate Identity)? & Kommunikationspolitik \\
\end{tabular}

Im Gegensatz dazu werden beim \bf{operativen Marketing} kurzfristige Entscheidungen getroffen, um auf Veränderungen am Markt zu reagieren. In diesen Bereich fallen daher fragen der folgenden Art:
\begin{tabular}{llr}
	1b & Mit welchen Produkten können kurzfristige Marktchancen wahrgenommen werden? & Produkt- \& Sortimentspolitik \\
	2b & Welche Werbemittel sollen eingesetzt werden? & Kommunikationspolitik \\
	3b & Mit welchen Preisen soll auf Marktsituationen reagiert werden? & Preispolitik \\
	4b & Welcher Absatzweg soll situations- und kundengerecht gewählt werden? & Distributionspolitik \\
\end{tabular}

\subsection{Aufgabenbereiche des Marketings}

Zu den \bf{Aufgabenbereichen} des Marketings gehören unter anderem die \bf{Marktforschung}, \bf{Produkt- und Sortimentspolitik} sowie die \bf{Preispolitik}, die auch Kontrahierungspolitik genannt wird. Der erste der drei Aufgabenbereiche deckt die Frage 1a des strategischen Marketings ab. Die Fragen 2a und 1b, welche Produkte angeboten werden sollen und mit welchen Produkten sich kurzfristige Marktchancen realisieren lassen, wird im Rahmen des zweiten Aufgabenbereiches \ql Produkt- und Sortimentspolitik\qr, beantwortet. Wie sich ein Unternehmen darstellen will, d.h. welche Coporate Identity es nach außen verkörpern will, wird zusammen mit der Frage 2b im Bereich \bf{Kommunikationspolitik} behandelt. Im Aufgabenbereich drei, der Preispolitik wird Frage 3b beantwortet und im Rahmen der \bf{Distributionspolitik} wird entschieden, welche Absatzwege gewählt werden soll (Frage 4b).

\subsection{Zuständigkeiten für Marketing}

Im Unternehmen übernimmt die Geschäftsleitung oder diKey-Account-Managere Vertriebsleitung Aufgaben des Marketings und bezieht bei Bedarf Agenturen hinzu. Eine Marketingabteilung soll die Kundenbeziehungen verbessern und den Absatz fördern. Product-Manager sind für die Produktkonzeption sowie Markterschließung der Produkte und Leistunge verantwortlich. Key-Account-Manager sollen die Geschäftsbeziehungen zu ihren Schlüsselkunden pflegen und neue Kunden gewinnen.

\subsection{Marktforschung}
\subsubsection{Grundbegriffe der Marktforschung}
Es wird zwischen Primär- (field research) und Sekundärforschung (desk research) unterschieden. Bei der Primärforschung werden wie der englische Begriff nahelegt im Feld, das heißt vor Ort neue Daten erhoben. Im Gegensatz versucht die Sekundärforschung aus bereits bestehenden Daten vom Schreibtisch aus neue Information zu gewinnen.

\noindent Die Primärforschung zeichnet sich dadurch aus, dass sie im Gegensatz zur Sekundärforschung teuer und aufwändig ist. Daher wird sie in der Regel von großen Unternehmen oder Instituten durchgeführt. Zu den Methoden der Primärforschung gehören das Panel, die Beobachtung, die Befragung und das Experiment. Bei einem Panel wird versucht, Veränderungen über einen bestimmten Zeitraum zu messen und stützt sich dabei auf die Analyse von Beobachtungen und Experiment. Beobachtungen können teilnehmend oder nicht-teilnehmend sein. Teilnehmende Beobachtungen kommen einem Experiment nah. Es wird nochmals zwischen Feld- und Laborbeobachtung unterschieden. Eine nicht-teilnehmende Beobachtung ist das, was normalerweise unter einer Beobachtung verstanden wird. Sekundärforschung bedient sich internen oder externen Informationsquellen. Interne Quellen sind solche, die bereits im Unternehmen vorhanden sind. Externe Quellen sind Informationen, die von anderen Unternehmen oder Institutionen bezogen werden. Darunter fallen unter anderem Statistiken, aber auch Artikel aus der Fachpresse.

Im Rahmen der Primär- und Sekundärforschung gibt es zudem die Instrumente Marktanalyse (sekundär), Marktbeobachtung (primär) und Marktprognose (sekundär). Eine Marktanalyse hat das Ziel das Angebot und die Nachfrage sowie die Konkurrenz zu ermitteln. Die Marktbeobachtung bezieht sich wie ein Panel auf einen Zeitraum und gibt dabei Informationen über die Entwicklung eines Marktes. Bei einer Marktprognose wird versucht, Aussagen über die zukünftige Entwicklung eines Marktes zu treffen. Dabei wird auf Analysen und Beobachtungen zurückgegriffen und bekannte Regelmäßigkeiten extrapoliert.

\subsubsection{Instrumente der Marktforschung}
Bei den Instrumenten der Marktforschung kann zwischen Marktuntersuchung und Erhebungsverfahren unterschieden werden. Welche Instrumente dies im einzelnen sind, ist den beiden Tabellen zu entnehmen.

%Tabelle: Marktuntersuchung
\noindent \begin{tabular}{|p{\dimexpr 0.20\linewidth-2\tabcolsep}|
				 p{\dimexpr 0.25\linewidth-2\tabcolsep}|
				 p{\dimexpr 0.25\linewidth-2\tabcolsep}|
				 p{\dimexpr 0.30\linewidth-2\tabcolsep}|} %4 Spalten
	\hline
	\multicolumn{4}{|c|}{{{\bf{\large Marktuntersuchung}}}} \\
	\hline
	\multicolumn{2}{|c|}{Maßnahmen} & Erläuterung & Beispiele \\
	\hline
%Markterkundung
Markterkundung & betriebsintern & {\bf unsystemische} Untersuchung der eigenen Informationen und Daten der Abteilungen Einkauf, Produktion und Verkauf & Mitarbeiter bringen ihre Erkenntnisse anhand von Reise- und Marktberichten oder durch Stellungnahmen in Besprechungen ein. \\
	\hline
%Marktforschung
Marktforschung nach dem Umfang der Untersuchung & Marktbeobachtung
& systematische, {\bf zeitraumbezogene} Untersuchung & Ein Marktforschungsinstitut wird beauftragt, systematisch einen Monat lang die Reaktion des Marktes auf verschieden Produktinserate zu untersuchen. \\
	\cline{2-4}
%Marktanalyse
& Marktanalyse & systematische, {\bf zeitpunktbezogene} Untersuchung & Es soll systematisch an einem Tag eine Kundenbefragung durchgeführt und ausgewertet werden. \\
	\hline
Marktforschung nach dem Ziel der Untersuchung
%Produktanalyse
& Produktanalyse & Systematische Untersuchung der Produkte hinsichtlich Preis-Leistungs-Verhältnis, Absatzchancen, Beratungsaufwand, Ertragschancen & Es soll herausgefunden werden, mit welchen Produkten man langfristig den Markt am besten bedienen und am meisten Gewinn erzielen kann. \\
%Konkurrenzforschung
	\cline{2-4}
& Konkurrenzforschung & Systematisch Untersuchung betrifft Produkte und -neuheiten sowie der Marktanteile und des Marktverhaltens der Mitbewerber. & Es soll systematisch untersucht werden, welche Mitbewerber einen Internetshop betreiben. \\
	\cline{2-4}
%Kundenanalyse
& Kundenanalyse, Bedarfs-, und Absatzforschung & Systematische Untersuchung des Marktes hinsichtlich des {\bf Gesamtbedarfs an Produkten} auf dem Markt sowie zu Kaufmotiven und Absatzmöglichkeiten. & Es wird eine Untersuchung in Auftrag gegeben, ob Schüler ein Interesse an einem Computerführerschein haben und wie groß der Markt sein wird. \\
	\hline
%Marktprognose
Marktprognose & Trendberechnung & Aufgrund von vorhandenen Marktdaten wird versucht, einen Trend festzustellen und eine Prognose in die Zukunft abzugeben. & Bei Handys wurde in den letzten Jahren ein Marktwachstum von über 10\% festgestellt. Dank iPhone wird in den nächsten fünf Jahren mit einer Absatzsteigerung von 60\% gerechnet. \\
	\hline
\end{tabular}

%Tabelle: Erhebungsverfahren
\noindent \begin{tabular}{|p{\dimexpr 0.2\linewidth-2\tabcolsep}|
				 p{\dimexpr 0.20\linewidth-2\tabcolsep}|
				 p{\dimexpr 0.60\linewidth-2\tabcolsep}|} %3 Spalten
\hline
\multicolumn{3}{|c|}{{{\bf{\large Erhebungsverfahren / -arten}}}} \\
\hline
Umfang
& Vollerhebung & Für das zu untersuchende Merkmal wird die {\bf gesamte Grundmenge} untersucht, z.B. werden alle Kunden befragt. \\
\cline{2-3}
& Teilerhebung & Es wird eine Stichprobe untersucht, wobei auf eine {\bf repräsentative Stichprobe} Wert gelegt werden sollte. Selektion erfolgt durch Zufallsauswahl (z.B. jeder 100. Kunde) oder per Quote nach festgelegten Merkmalen (z.B. Alter: 40\% unter 20 Jahre). \\
\hline
Primärerhebung
& Befragung & schriftlich, mündlich, telefonisch, online anhand vorgegebener Fragen: Einfachauswahlfragen, Mehrfachauswahlfragen, Skalenfragen, Maßzahlen oder freie Fragen. \\
\cline{2-3}
& Interview & persönlich nach {\bf vorgegebnen Merkmalen} oder Fragen; Vorteile: Erklärungen, Nachfragen möglich, individuell. \\
\cline{2-3}
& Beobachtung & Dauerbeobachtung oder mehrere Kurzbeobachtungen als Multimomentbeobachtung (kostengünstiger) zu einem bestimmten Beobachtungszweck. \\
\cline{2-3}
& Panel & {\bf Regelmäßige Befragung} einer bestimmten Personengruppe erfolgt über einen längeren Zeitraum mit den gleichen Fragen. \\
\cline{2-3}
& Test/Experiment & Meinungserhebung aus einer (neutral gestellten) Testumgebung, z.B. werden Kunden neutral verpackte Produkte oder Warenproben zur Begutachtung vorgelegt oder zum Test oder zur Erprobung zur Verfügung gestellt. \\
\hline
Sekundärerhebung & Schon vorliegende Daten werden verwendet
& interne Daten, z.B. werden Kundendaten, Umsatzdaten, Messberichte, Reklamationen ausgewertet \\
\cline{3-3}
& & externe Daten, z.B. werden Daten der statistischen Ämter, der Kammer und Innungen, Verbände, Zeitschriften usw. ausgewertet \\
\hline
\end{tabular}






\subsection{Vollkostenrechnung}
\subsubsection{Vorwärtskalkulation}
\subsubsection{Rückwärtskalkulation}

\subsection{Produkt- und Sortimentspolitik}


\subsubsection{Aspekte}

\begin{tabular}{lll}
	& Herstellung & Handel \\
	\hline
	Maßnahmen & {\bf Erw. d. Produktprog.} & {\bf Sortimentserw.} \\
	Zusammensetzung & ständig vs zeitweise & Kernsort. vs Randsort. \\
	{Struktur/\\Umfang} & \multicolumn{2}{l}{{\bf breit vs schmal und tief vs flach}} \\

\end{tabular}

\subsubsection{Maßnahmen}

\begin{itemize}
	\item {\bf Produktgestaltung}: (1) Qualität, (2) Aufmachung, (3) Verpackung, (4) Markierung
	\item {\bf begleitende Servicepolitik}: (1) Kundendienst, (2) Garantie, (3) Schulung
	\item {\bf prozessorientiert Produktpolitik}: (1) Innovation, (2) Differenzierung, (3) Diversifikation, (4) Variation, (5) Elimination
\end{itemize}

\subsubsection{Produktlebensphasen}

Es werden fünf Phasen im Leben eines Produktes unterschieden. Der Grundgedanke dabei ist, dass jedes Produkt, jede Produktgruppe und jede Produktklasse dieselben Phasen durchlaufen. Der Aussagewert ist entsprechend begrenzt, weil es sich schon im Ansatz um eine Idealisierung handelt. Daher dient es eher als idealtypische Veranschaulichung.

\begin{tabular}{llll}
	Phasen & Kapitalbedarf & Werbekosten & Rentabilität \\
	\hline
	Einführung & hoch & hoch & negativ \\
	Wachstum & mittel & hoch & positiv \\
	Reife & niedirg & mittel & sehr positiv \\
	Sättigung & gering & gering & sehr positiv \\
	Degeneration & gering & kaum & fallend \\
\end{tabular}

Der Übergang von der Einführungsphase in die Wachstumsphase findet am Point-of-Break-Even statt.


Jede Phase zeichnet sich durch eigenen Probleme und Stragien aus. 

\subsection{Kommunikationspolitik}


\subsection{Preis- und Konditionenpolitik}

\subsubsection{Produktpolitik: Preisbildung und Preisstrategien}

Im Rahmen der Kontrahierungspolitik (Preis- und Konditionenpolitik) finden alle Entscheidungen statt, die den Preis betreffen. Die Preisgestaltung kann auf verschiedene Weisen erfolgen.

\subsubsection{Preisgestaltung}

Bei der Preisgestaltung werden drei Faktoren unterschieden. (1) Kostenorientierung, (2) Nachfrageorientierung und (3) Konkurrenzorientierung. Bei dem ersten Faktor wird der Preis anhand der anfallenden Kosten ermittelt. Durch Zuschlagskalkulation wird der Angebots- bzw. Verkaufspreis ermittelt. Der Preis wird \qr\ from company to market\qr\ ermittelt. Handlungskosten sind alle Kosten, die bei einem Händler neben den Einstandspreisen der verkauften Waren regelmäßig anfallen. Durch Handlungskostenzuschlag sollen alle Kosten des Händlers gedeckt werden. Der Gewinnzuschlagssatz ist das Ergebnis von Erfahrungswerten. Die kostenorientierte Preisbildung ist mit einigen Problemen verbunden. Zum einen kann es sein, dass ein kalkulierter Preis sich als marktfern herausstellt und zum anderen könnte die Zahlungsbereitschaft des Kunden nicht vollständig ausgeschöpft werden, d.h. der Kunde wäre bereit gewesen mehr zu bezahlen.

\begin{itemize}
	\item Preisführer
	\item Preisfolger
	\item Preiskämpfer
\end{itemize}

\begin{itemize}
	\item Skimming (Abschöpfung):
	\item Penetration (Marktdurchdringung):
\end{itemize}

\subsubsection{Preisdifferenzierung}

Eine Preisdifferenzierung liegt vor, wenn ein Unternehmen für gleiche oder gleichartige Produkte unterschiedliche Preise verlangt, die sich nicht oder nicht gänzlich durch Qualitätsunterschiede begründen lassen. Im Rahmen der Preispolitik lassen sich Preise anhand von vier Kategorien differenzieren:

\begin{enumerate}
	\item {\bf räumlich}: Veräußerung von Waren auf regional abgegrenzten Märkten zu verschieden hohen Preisen, z.B. Preisdifferenzierung  zwischen In- und Ausland.
	\item {\bf zeitlich}: Forderung verschieden hoher Preise für gleichartige Waren je nach der zeitlichen Nachfrage, z.B. Verleih von Kinofilmen.
	\item {\bf sachlich}: Preishöhe je nach dem Verwendungszweck der Produkte, z.B. verschiedene Strom- und Gastarife für Industrie- und Haushaltsverbrauch.
	\item {\bf persönlich}: Preisstellung je nach der marketingpolitischen Bedeutung (z.B. A- oder C-Kunden) und/oder den Absatzfunktionen der Zielgruppen, z.B. Groß- oder Einzelhandel.
\end{enumerate}

\subsubsection{Preiselastizität}

$Preiselastizität = (-1)\times(\frac{neue Menge - alte Menge}{alte Menge} / \frac{neuer Preis - alter Preis}{alter Preis})$\\
Die Preiselastizität wird mit $-1$ multipliziert, damit das Ergebnis positiv wird. Werte unter $1$ gelten als {\bf unelastisch} und Werte über $1$ entsprechend als {\bf elastisch}. Beispiele dafür sind Güter des Grundbedarfs ($<1$) und Güter des gehobenen Bedarfs wie beispielsweise Autos oder Unterhaltungselektronik ($>1$). Wenn die Elastizität $1$ beträgt, spricht man von {\bf isoelastischer Nachfrage}. Diese kommt in der Realität sehr selten vor. Sollte die Preiselastizität $0$ betragen, bedeutet dies eine {\bf vollkommen unelastische Nachfrage}. Ein Beispiel hierfür sind Medikamente. Nur weil ein Medikamet billiger wird, wird es nicht zwangsläufig häufiger genommen.

\subsubsection{Analysen}

\begin{itemize}
	\item Zielgruppenanalyse
	\item Portfolioanalyse
	\item Produktlebenszyklus-Analyse
	\item Produktprogrammstruktur-Analyse
	\item Geschäftsfeld-Analyse
\end{itemize}

\subsubsection{Zielgruppenanalyse}

In der Zielgruppenanalyse werden soziodemographische und psychographische Merkmale erfasst, um verschiedene Zielgruppen voneinander zu unterscheiden.

\subsubsection{Produktpositionierung}

Die Produktpositionierung entspricht der unverwechselbaren Kennzeichnung relevanter Eigenschaften eines Produktes in den Köpfen der Verbraucher. Die Positionierung orientiert sich an den wahrgenommenen Leistungsmerkmalen. Zu den {\bf Kernelementen} der klassichen Produktpositionierung gehören die (1) {\bf Datenerhebung}, (2) {\bf Verdichtung der Daten} zu Grafiken, (3) die {\bf \textsc{Ist}-Analyse} der eigenen Position und der der Konkurrenz, (4) die Ermittelung der {\bf Idealposition} aus Kudnensicht und (5) die der {\bf Distanz zwischen \textsc{Ist} und Ideal.

{\bf Beispiel}: Neupositionierung einer Hotelkette.
\begin{enumerate}
	\item Marktsegmentierung anhand von Segmentierungskriterien. Segmentierungskriterien sind Kunden- sowie Bedürfnismerkmale. In diesem Fall verläuft die Segmentierung beispielsweise zwischen Privat- und Geschäftskunden , der vorhandenen Kaufkraft und dem Alter der Kundschaft.
	\item Ermittlung relevanter Kriterien für Hotelwahl. Anhand von Sekundärmaterials und Befragungen lässt sich das Nachfrageverhalten der anvisierten Kunden ermitteln. Für das Beispiel bedeutet dies, dass ermittelt werden kann, welche allgemeine Ausstattung im Hotel erwartet wird; bspw.: Tagungsräume, Konferenztechnik, Telekommunikationstechnik, Gastronomiemöglichkeiten, Freizeitmöglichkeiten
	\item Gewichtung der Kriterien (meist Vierfelder-Matrix)
	\item Bewertung der Konkurrenz 
	\item Verdichtung bzw. Reduzierung der Kriterien
	\item Einordnung des eigenen Hotels und der Konkurrenz bezüglich der ermittelten Dimensionen (Vierfelder-Matrix)
\end{enumerate}	 
	 
\subsubsection{Portfolioanalyse}

Im Rahmen der Portfolioanalyse werden strategische Geschäftseinheiten respektive die Produktlinien in einer Vier- oder Mehrfelder-Matrix verortet. Die Portfolioanalyse wird auch Geschäftsfeld-Analyse genannt.

\paragraph{Boston-Consulting-Group: Marktwachstum-Marktanteil-Portfolio}~\\

	- Boston-Consulting-Group: Marktwachstum-Marktanteil-Portfolio
		- vergleicht Marktwachstum mit relativem Marktanteil
		* jeweils hoch und niedrig
					Marktanteil	/ Marktwachstum
			* Stars			hoch	/ hoch
			 - beanspruchen viele Ressourcen und erwirtschaften
			   kaum Überschüsse
			* Cash Cows		hoch	/ niedrig
			 - erfolgreiche und etablierte Produkte
			 - sichern kurzfristig den Erfolg des Unternehmens
			* Question Marks	niedrig	/ hoch
			 - binden finanzielle Mittel, Zukunft ungewiss
			* Poor Dogs		niedrig	/ niedrig

		- Kreisposition entspr Verortung, Größe entspr Bedeutung
	- McKinsey kritisiert Eindimensionalität
	
\paragraph{McKinsey: Marktattraktivität gegenüber relativem Wettbewerbsvorteil}
		- stattdessen Marktattraktivität vs rel. Wettbewerbsvorteil
			- Marktattraktivität: bestimmt durch
			 * Marktwachstum
			 * Marktgröße
			 * Marktqualität
			 * Umweltsituation
			- relative Wettbewerbsvorteile
			 * Größe der relativen Marktposition
			 * relatives Produktionspotenzial
			 * relative Personalqualität
		- je niedrig, mittel, hoch

\subsubsection{Produktlebenszyklus-Analyse}
	- Grundidee: Produkte, P.Gruppen und P.Klassen durchlaufen dieselben
	 Phasen; Umsatz rel zu Gewinn -> 5 Phasen
		- Einführungsphase
			 Beginnt mit Release, endet mit erreichen der
			 Gewinnschwelle (Entwicklung und Marketing müssen sich
			 erst amortisieren). Wenn neuartiges Produkt markt-
			 beherrschend, dann Preiselastizität gering (iPhone)
				Kunden: Aufgeschlossen, innovationswillig
				- Innovatoren
		- Wachstumsphase
			 Bekanntheit nimmt zu, Produkt erwirtschaftet Gewinn
			 me-too-Produkte erscheinen, Preiselastizität nimmt zu
			 Bekanntheit/Festigung des Images: mehr Kunden
				Kunden: erste Stammkunden, Early Adopter
		- Reifephase
			 Wachstumsrate und Gewinn gehen zurück
			 * Umsatz immer noch steigend
			 Preiselastizität der Nachfrage nimmt stark zu,
			 Preispolitik wird effektives absatzpolitisches Instr
			 Steigende Anzahl der Konkurrenten, Differenzierung
				Kunden: konservative Einstellung,
					frühe Mehrheit
		- Sättigungsphase
			 Beginnt, wenn Umsatz nicht mehr wächst, Nachfrage
			 stagniert, Preiselastizität der Nachfrage ist hier am
			 größten
				Kunden: späte Mehrheit
		- Degenerationsphase
			 Bedürfnis wird durch andere Produkte befriedigt
			 Nachfrage sinkt rapide, Umsatz sinkt, Verluste
			 Preiselastizität nähert sich dem Nullpunkt
				Kunden: Nachzügler 
	- Aussagewert begrenzt, da Idealisierung
		- Flop: schnelles Wachstum, schneller Rückgang
		- Relaunch

\subsubsection{Produktprogrammstruktur-Analyse}
	- vergleich Altersstruktur, Umsatzstruktur, Kundenstruktur,
	 Deckungsbeitragsstruktur und Struktur der Geschäftsfelder
		- Altersstruktur: vergleicht Produktlebenszyklus
		- Umsatzstruktur: Umsatzprofile - welches Produkt
		 erwirtschaftet wie viel Prozent des Gewinns (Lorenz-Kurve)
		- Kundenstruktur: ABC-Analyse und Einteilung der Kunden
		 Einteilung: Umsatz, Gewinn, stragetische Bedeutung
		- Deckungsbeitrag: Anteil des Produktes am Erfolg des
		 Unternehmens (Deckungsbeitragsrechnung)

\subsubsection{Produktinnovation}

Produktinnovationen sind teuer und ressourcenintensiv. Daher ist sorgfältige Planung notwendig. Gründe, die Produktinnovationen notwendig machen, sind beispielsweise \dots
\begin{itemize}
	\item Wachstumsstrategie
	\item Produkt ist veraltet
	\item Patente laufen aus
	\item Produkt-Mix ist einseitig
	\item technische Neuerungen
	\item geänderte rechtliche Restriktionen
	\item veränderte Kundenansprüche
	\item unternehmensinterne Restriktionen
	\item Zufallsentwicklungen
\end{itemize}

Im Rahmen der Produktinnovation wird zwischen Produktdifferenzierung und Produktdiversifizierung unterschieden. {\bf Produktdifferenzierung} bedeutet, dass zu den vorhandenen Produkten zusätzliche Produkte entwickelt und vermarktet werden (Produktvariationen). Dieses Vorgehen wird auch als {\bf Marktsegmentierung} bezeichnet. Marktsegmentierung bezeichnet die Aufteilung eines Gesamtmarktes in hinsichtlich ihrer Marktreaktionen intern homogenener und untereinander heterogener Untergruppen. Gründe dafür können sein:

\begin{itemize}
	\item Eroberng neuer Märkte mit bekannten Produkten
	\item konsequente Marktsegmentierung
	\item Ausnutzung von Synergieeffekten
	\item technischer Fortschritt
	\item Anpassung an Mode
	\item {rechtliche Unterschiede in Ländern}
	\item Einführung eines erfolgreichen Nischenproduktes
\end{itemize}

{\bf Produktdiversifizierung} bezeichnet im Gegensatz die Entwicklung und Vermarktung neuer Produkte. Innerhalb der Produktdiversifikation wird zwischen {\bf horizontaler}, {\bf vertikaler} und {\bf lateraler} Diversifikation unterschieden. Dabei bezeichnet das erstere die Entwicklung von Produkten derselben Wirtschaftsstufe, das zweite die von Produkten höherer oder niedrigerer Stufe und letzteres die Entwicklung von sachlich unzusammenhängenden Produkten.
			 
			 
\subsection{Distributionspolitik}
