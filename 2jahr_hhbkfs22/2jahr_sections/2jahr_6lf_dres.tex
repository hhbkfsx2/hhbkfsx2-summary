\section{Lernfeld 6: Programmieren - C-Sharp}

\subsection{Objektorientierung}

Ein Objekt ist eine Instanz einer Klasse. Ein Objekt hat Attribute und beherrscht Methoden. Attribute sind Konstanten oder Variablen mit einem festen Datentyp. In der Regel sind Attribute private, dass bedeutet, dass sie von außen nicht verändert werden können (Datenkapselung). Daher gibt es zum Ändern von Attributen meist public Methoden, die entsprechende Abfragen enthalten, damit Attribute nur korrekte Daten erhalten. Methoden sind entweder Funktionen oder Prozeduren. Im Gegensatz zu Funktionen liefern Prozeduren keinen Zückgabewert.

\subsection{Übung: Rennwagen}

\subsection{Übung: Kleines 1x1}

Bei der Übung \ql Kleines 1x1\qr\ soll der unten gezeigte Output auf der Konsole ausgegeben werden. Anhand dieser Übung sollen erste Unterschiede zwischen PHP und C\# aufgezeigt werden.

\begin{lstlisting}
01 02 03 04 05 06 07 08 09 10 
02 04 06 08 10 12 14 16 18 20 
03 06 09 12 15 18 21 24 27 30 
04 08 12 16 20 24 28 32 36 40 
05 10 15 20 25 30 35 40 45 50 
06 12 18 24 30 36 42 48 54 60 
07 14 21 28 35 42 49 56 63 70 
08 16 24 32 40 48 56 64 72 80 
09 18 27 36 45 54 63 72 81 90 
10 20 30 40 50 60 70 80 90 100
\end{lstlisting}

% PHP-Lösung
\lstinputlisting
	[caption={Lösung der Übung mit PHP}
	\label{lst:6lf_dres_kleines1x1.php},
	basicstyle=\small,
	captionpos=b,
	frame=single,
	numbers=left,
	numberstyle=\small,
	numberblanklines,
	numbersep=10pt,
	xleftmargin=15pt]
	{2jahr_code/2jahr_6lf_dres_kleines1x1.php}
\newpage
% C#-Lösung
\lstinputlisting
	[caption={Lösung der Übung mit C\#}
	\label{lst:6lf_dres_kleines1x1.cs},
	basicstyle=\small,
	captionpos=b,
	frame=single,
	numbers=left,
	numberstyle=\small,
	numberblanklines,
	numbersep=10pt,
	xleftmargin=15pt]
	{2jahr_code/2jahr_6lf_dres_kleines1x1.cs}