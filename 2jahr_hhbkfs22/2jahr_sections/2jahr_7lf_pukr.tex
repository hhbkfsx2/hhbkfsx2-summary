\section{Lernfeld 7 - Vernetzte IT-Systeme / PUKR} % Pukropski

%%% Anfang: Was ist das Internet?
\subsection{Was ist das Internet?}

%%% Ende: Was ist das Internet?
%%%%%%%%%%%%%%%%%%%%%%%%%%%%%%%%%%%%%%%%%%%%%%%%%%%%%%%%%%%%%%%%%%%%%%%%%%%%%%

%%% Anfang: DNS - Domain Name System
\subsection{DNS - Domain Name System}

%%% Ende: DNS - Domain Name System
%%%%%%%%%%%%%%%%%%%%%%%%%%%%%%%%%%%%%%%%%%%%%%%%%%%%%%%%%%%%%%%%%%%%%%%%%%%%%%

%%% Anfang: IPv4 - Internet Protocol
\subsection{IPv4 - Internet Protocol}

\subsubsection{Classful Subnetting}

Beim classful subnetting gibt es grundsätzlich drei Netzklassen: A, B und C. Diese zeichnen sich durch ihre Subnetzmasken aus, 255.0.0.0, 255.255.0.0 und 255.255.255.0. Beim classful subnetting wird zwischen der Host-ID und der Net-ID unterschieden. Die beiden IDs lassen sich durch die Subnetzmaske aus der IP herleiten.

% Beispiel IP, Subnetzmaske, Net-ID und Host-ID
\lstinputlisting
	[caption={Beispiel zu Classful Networking}
	\label{lst:7lf_classful.example},
	basicstyle=\small,
	captionpos=b]
	{2jahr_code/2jahr_7lf_classful.example.txt}

\subsubsection{Classless Inter-Domain Routing}

Im Gegensatz zum classful subnetting zeichnet sich CIDR (Classless Inter-Domain Routing) dadurch aus, dass statt drei Netzklassen, wird die Subnetzmaske in Form eines Suffix {\it /XX} dargestellt. Das Suffix gibt dabei an, wie viele der Bits von links beginnend auf 1 gesetzt sind. Sobald die Subnetzmaske bekannt ist, lassen sich dadurch Host- und Net-ID ermitteln.

% Beispiel CIDR IP und Subnetzmaske
\lstinputlisting
	[caption={Beispiel zu CIDR}
	\label{lst:7lf_classless.example},
	basicstyle=\small,
	captionpos=b]
	{2jahr_code/2jahr_7lf_classless.example.txt}
	
%%% Ende: IPv4 - Internet Protocol
%%%%%%%%%%%%%%%%%%%%%%%%%%%%%%%%%%%%%%%%%%%%%%%%%%%%%%%%%%%%%%%%%%%%%%%%%%%%%%