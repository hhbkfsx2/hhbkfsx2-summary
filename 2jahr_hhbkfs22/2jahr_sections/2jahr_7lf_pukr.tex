\section{Lernfeld 7 - Vernetzte IT-Systeme / PUKR} % Pukropski

%%% Anfang: Was ist das Internet?
\subsection{Was ist das Internet?}

Das Internet ist ein Netzwerk von Netzwerken. Das Internet besteht aus der Vernetzung von sogenannten Autonomen Systemen (AS). Jedes AS hat eine mit IP vergleichbare Adresse, die AS Number (ASN).

%%% Ende: Was ist das Internet?
%%%%%%%%%%%%%%%%%%%%%%%%%%%%%%%%%%%%%%%%%%%%%%%%%%%%%%%%%%%%%%%%%%%%%%%%%%%%%%

%%% Anfang: DNS - Domain Name System
\subsection{DNS - Domain Name System}

%%% Ende: DNS - Domain Name System
%%%%%%%%%%%%%%%%%%%%%%%%%%%%%%%%%%%%%%%%%%%%%%%%%%%%%%%%%%%%%%%%%%%%%%%%%%%%%%

%%% Anfang: IPv4 - Internet Protocol
\subsection{IPv4 - Internet Protocol}

\subsubsection{Classful Subnetting}

Beim classful subnetting gibt es grundsätzlich drei Netzklassen: A, B und C. Diese zeichnen sich durch ihre Subnetzmasken aus, 255.0.0.0, 255.255.0.0 und 255.255.255.0. Beim classful subnetting wird zwischen der Host-ID und der Net-ID unterschieden. Die beiden IDs lassen sich durch die Subnetzmaske aus der IP herleiten.

% Beispiel IP, Subnetzmaske, Net-ID und Host-ID
\lstinputlisting
	[caption={Beispiel zu Classful Networking}
	\label{lst:7lf_classful.example},
	basicstyle=\small,
	captionpos=b]
	{2jahr_code/2jahr_7lf_classful.example.txt}

\subsubsection{Classless Inter-Domain Routing}

Im Gegensatz zum classful subnetting zeichnet sich CIDR (Classless Inter-Domain Routing) dadurch aus, dass statt drei Netzklassen, wird die Subnetzmaske in Form eines Suffix {\it /XX} dargestellt. Das Suffix gibt dabei an, wie viele der Bits von links beginnend auf 1 gesetzt sind. Sobald die Subnetzmaske bekannt ist, lassen sich dadurch Host- und Net-ID ermitteln.

% Beispiel CIDR IP und Subnetzmaske
\lstinputlisting
	[caption={Beispiel zu CIDR}
	\label{lst:7lf_classless.example},
	basicstyle=\small,
	captionpos=b]
	{2jahr_code/2jahr_7lf_classless.example.txt}
	
%%% Ende: IPv4 - Internet Protocol
%%%%%%%%%%%%%%%%%%%%%%%%%%%%%%%%%%%%%%%%%%%%%%%%%%%%%%%%%%%%%%%%%%%%%%%%%%%%%%

%%%%%%%%%%%%%%%%%%%%%%%%%%%%%%%%%%%%%%%%%%%%%%%%%%%%%%%%%%%%%%%%%%%%%%%%%%%%%%
%%% Anfang: IPv6
\subsection{IPv6}


%%% Ende: IPv6
%%%%%%%%%%%%%%%%%%%%%%%%%%%%%%%%%%%%%%%%%%%%%%%%%%%%%%%%%%%%%%%%%%%%%%%%%%%%%%

%%%%%%%%%%%%%%%%%%%%%%%%%%%%%%%%%%%%%%%%%%%%%%%%%%%%%%%%%%%%%%%%%%%%%%%%%%%%%%
%%% Anfang: Transportschicht

\subsection{Transportschicht}

%%% Ende: Transportschicht
%%%%%%%%%%%%%%%%%%%%%%%%%%%%%%%%%%%%%%%%%%%%%%%%%%%%%%%%%%%%%%%%%%%%%%%%%%%%%%


%%%%%%%%%%%%%%%%%%%%%%%%%%%%%%%%%%%%%%%%%%%%%%%%%%%%%%%%%%%%%%%%%%%%%%%%%%%%%%
%%% Anfang: Router und Routingprotokolle
\subsection{Router und Routingprotokolle}

Routingprotokolle werden dazu verwendet, um Routing-Entscheidung zu automatisieren

\subsubsection{RIP und OSPF}

Route Information Protocol (RIP) und Open Shortest Path First (OSPF) ermitteln beide die kürzesten Wege innerhalb eines Netzwerkes, wobei \ql kürzeste\qr\ bei RIP bedeutet, dass die Anzahl der Hops möglichst gering ist und bei OSPF weitere Metriken angesetzt werden können, um die Routing-Entscheidungen zu steuern.

\subsubsection{Border Gateway Protocol - BGP}

Das Border Gateway Protocol wird im Internet dazu verwendet, die IP-Adressen, die ein AS anbietet, zu announcen und entsprechend Routing-Entscheidungen anhand der Announcments zu treffen.

\subsection{Algorithmen}

%%% Ende: Router und Routingprotokolle
%%%%%%%%%%%%%%%%%%%%%%%%%%%%%%%%%%%%%%%%%%%%%%%%%%%%%%%%%%%%%%%%%%%%%%%%%%%%%%


\subsection{Anwendungsschicht}

\subsection{Netzwerksicherheit und Firewalls}

\subsubsection{Firewalls: Zonen}

Beim Firewalling werden generell drei Zonen unterschieden. Das Intranet bzw. die Trusted Zone, die Demilitarisierte Zone (DMZ) und das Internet, auch Untrusted Zone genannt.

\paragraph{Trusted Zone}~\\
\paragraph{DMZ}~\\

Die Demilitarisierte Zone (DMZ) wird auch als \ql perimeter network\qr\ bezeichnet.

\paragraph{Untrusted Zone}~\\

\subsubsection{Firewalls: Generationen}

Aktuell werden drei Generationen von Firewalls unterschieden. Zum erst sind das die klassischen stateless firewalls, zum zweiten stateful firewalls und zu letzt die application layer firewalls.

\subsubsection{Firewalls: Konzepte}

Einstufige Firewall:

Zweistufige Firewall: Bei einer zweistufigen Firewall wird zwischen Intranet und DMZ sowie zwischen DMZ und Internet eine Firewall plaziert. In der Praxis werden zwei Firewalls verschiedener Hersteller verwendet, damit ein Sicherheitsproblem bei einer Firewall nicht direkt das Intranet öffnet.

Mehrstufige Firewall:
