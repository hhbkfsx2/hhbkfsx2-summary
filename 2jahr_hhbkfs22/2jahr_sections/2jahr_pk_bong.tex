\section{Politik}

\subsection{Sozialversicherungen}

Ein Sozialstaat ist ein Staat, der versucht, seine Bürger durch ein System von Sozialleistungen in finanziellen Notsituationen vor dem Abgleiten in die Armut zu schützen. In Deutschland besteht dieses System von Sozialleistungen vor nämlich aus einer Reihe von Sozialversicherungen.

\subsubsection{Meilensteine der Sozialpolitik}

\paragraph{1863 - Ferdinand Lassalle}
\paragraph{1878 - Bismarks Sozialistengesetz}
\paragraph{1881 - Bismark'sche Sozialgesetze}
\paragraph{1919 - soziale Grundrechte}
\paragraph{1933 - Nazionalsozialismus}
\paragraph{1949 - Unterschiede zwischen Ost und West}
\paragraph{1990 - Wiedervereinigung} 
\paragraph{2003 - Agenda 2010}
\paragraph{2014 - Soziales Gefälle und Maßnahmen}

\subsubsection{Krankeversicherung}
\subsubsection{Arbeitslosenversicherung}
\subsubsection{Unfallversicherung}
\subsubsection{Rentenversicherung}
\subsubsection{Pflegeversicherung}

\subsection{Tarifverhandlungen}

In der Praxis gewähren Arbeitgeber auch nicht organisierten Mitarbeitern dieselben Leistungen wie Gewerkschaftsmitgliedern.

Die Tarifautonomie wird durch das Grundgesetzt geschützt (GG Art. 9 Abs. 3).

Tarifverträge können sogenannte \bf{Öffnungsklauseln} enthalten, die einem Arbeitgeber im Austausch für Gegenleistungen erlauben, in bestimmten Fällen von den tariflichen Mindestforderungen abzuweichen. Eine mögliche Gegenleistung für die Inanspruchnahme dieser Klauseln ist der Verzicht auf betriebsbedingte Kündigungen.

\subsubsection{Funktionen von Tarifverträgen}

Schutzfunktion
Friedensfunktion (Friedenspflicht)
Ordnungsfunktion
Richtlinienfunktion

\subsubsection{Arten von Tarifverträgen}

Manteltarifvertrag
Rahmentarifvertrag
Lohn-, Gehalts- und Entgelttarifvertrag
Einzeltarifvertrag

Firmentarifvertrag wird mit einem einzelnen Unternehmen geschlossen und auch als Haustarifvertrag bezeichnet. Bei der Unterteilung nach räumlicher Geltung ist der Flächentarifvertrag der häufigste. 


\subsubsection{Geltung von Tarifverträgen}

In einem Tarifvertrag wird unter anderem auch festgelegt, für wen der jeweilige Tarifvertrag gilt. Dies wird festgelegt durch vier Dimensionen: (1) Räumliche Geltung, (2) Fachliche Geltung, (3) Persönliche Geltung und (4) Zeitliche Geltung. Damit werden die Fragen 'Wo gilt der Vertrag?', 'Für welche Branchen?', 'Für welche Gruppen?' und 'Für wie lange?' abgedeckt. Nach Ablauf der zeitlichen Geltung bleibt ein Vertrag bis zum Abschluss eines neuen Vertrages weiterhin gültig (sog. 'Nachhaltigkeit').

\subsubsection{Allgemeinverbindlichkeit}

Tarifverträge können vom Bundesministerium für Arbeit und Sozialordnung im Einvernehmen mit dem Tarifausschuss für allgemeien verbindlich erklärt werden. Voraussetzungen dafür sind zum ersten, dass ein Tarifpartner die Allgemeinverbindlichkeit beantragt, zum zweiten, dass der tarifgebundene Arbeitgeber mindestens 50\% der betroffenen Arbeitnehmer beschäftigt und drittens die Allgemeinverbindlichkeit geboten erscheint. 

\subsubsection{Ablauf von Tarifverhandlungen}

\begin{enumerate}
	\item Der alte Vertrag läuft ab oder wird gekündigt.
	\item Der Arbeitgeberverand lehnt Vorderungen ab.
	\item Die Gewerkschaft lehnt das Gegenangebot ab.
	\item Ergebnislose Verhandlungen münden in Warnstreiks.
	\item Am Ende des Arbeitskampfes steht ein Kompromis.
\end{enumerate}

\subsubsection{Glossar 'Tarifverhandlungen'}

\subsection{[3. Block]}
\subsection{[4. Block]}