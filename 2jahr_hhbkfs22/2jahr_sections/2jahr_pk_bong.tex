\section{Politik}

\subsection{Sozialversicherungen}

Ein Sozialstaat ist ein Staat, der versucht, seine Bürger durch ein System von Sozialleistungen in finanziellen Notsituationen vor dem Abgleiten in die Armut zu schützen. In Deutschland besteht dieses System von Sozialleistungen vor nämlich aus einer Reihe von Sozialversicherungen.

\subsubsection{Meilensteine der Sozialpolitik}

\paragraph{1863 - Ferdinand Lassalle}
\paragraph{1878 - Bismarks Sozialistengesetz}
\paragraph{1881 - Bismark'sche Sozialgesetze}
\paragraph{1919 - soziale Grundrechte}
\paragraph{1933 - Nazionalsozialismus}
\paragraph{1949 - Unterschiede zwischen Ost und West}
\paragraph{1990 - Wiedervereinigung} 
\paragraph{2003 - Agenda 2010}
\paragraph{2014 - Soziales Gefälle und Maßnahmen}

\subsubsection{Krankeversicherung}
\subsubsection{Arbeitslosenversicherung}
\subsubsection{Unfallversicherung}
\subsubsection{Rentenversicherung}
\subsubsection{Pflegeversicherung}

\subsection{[2. Block]}
\subsection{[3. Block]}
\subsection{[4. Block]}