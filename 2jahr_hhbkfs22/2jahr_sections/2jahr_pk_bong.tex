\section{Politik}

\subsection{Sozialversicherungen}

Ein Sozialstaat ist ein Staat, der versucht, seine Bürger durch ein System von Sozialleistungen in finanziellen Notsituationen vor dem Abgleiten in die Armut zu schützen. In Deutschland besteht dieses System von Sozialleistungen vor nämlich aus einer Reihe von Sozialversicherungen.

\subsubsection{Meilensteine der Sozialpolitik}

\paragraph{1863 - Ferdinand Lassalle}
\paragraph{1878 - Bismarks Sozialistengesetz}
\paragraph{1881 - Bismark'sche Sozialgesetze}
\paragraph{1919 - soziale Grundrechte}
\paragraph{1933 - Nazionalsozialismus}
\paragraph{1949 - Unterschiede zwischen Ost und West}
\paragraph{1990 - Wiedervereinigung} 
\paragraph{2003 - Agenda 2010}
\paragraph{2014 - Soziales Gefälle und Maßnahmen}

\subsubsection{Krankeversicherung}
\subsubsection{Arbeitslosenversicherung}
\subsubsection{Unfallversicherung}
\subsubsection{Rentenversicherung}
\subsubsection{Pflegeversicherung}

\subsection{Tarifverhandlungen}

Wenn ein Tarifvertrag abgelaufen ist oder gekündigt wurde, werden neue Verhandlungen geführt. Dazu wird eine Kommission benannt, die aus Interessenvertretern der Tarifparteien, also der Gewerkschaft und des Arbeitgeberbundes, bestehen. Beide verfolgen dabei ihre eigenen, oft gegenläufigen Ziele. Die Ziele bzw. Interessen der Arbeitnehmer sind möglicherweise Arbeitsplatzsicherung, Lohn-/Gehaltserhöhung und/oder bessere Arbeitsbedingungen. Mögliche Interessen der Arbeitgeber sind die Sicherung der Wettbewerksfähigkeit, an der Produktivität orientierte Löhne und/oder Gewinnsteigerung. 

Die sogenannte Tarifautonomie wird durch das Grundgesetz garantiert (GG Art. 9 Abs. 3). Das bedeutet, dass die Tarifpartner ohne Einmischung des Staates Tarifverträge aushandeln dürfen. Damit üben Gewerkschaften und Arbeitgeberverbände ein Stück öffentlicher Rechtsgewalt zur Schaffung rechtsverbindlicher Ordnung für große Bereiche des Arbeitslebens aus.

Tarifverträge können sogenannte {\bf Öffnungsklauseln} enthalten, die einem Arbeitgeber im Austausch für Gegenleistungen erlauben, in bestimmten Fällen von den tariflichen Mindestforderungen abzuweichen. Eine mögliche Gegenleistung für die Inanspruchnahme dieser Klauseln ist der Verzicht auf betriebsbedingte Kündigungen.

Tarifverträge werden neu verhandelt, um beispielsweise die Entwicklung des Gehaltes an die Gegebenheiten anzupassen. Die Entwicklung des Gehaltes sollte zumindest einen Inflationsausgleich bieten und mit Hinblick auf die wirtschaftliche Lage des Unternehmens eine Teilhabe an der Produktivitäts- und Gewinnsteigerung abbilden.

Der Organisationsgrad gibt an, wie viel Prozent der Arbeitnehmer eines Unternehmens gewerkschaftlich organisiert sind.

\subsubsection{Funktionen von Tarifverträgen}

	\begin{itemize}
		\item Schutzfunktion: Der Arbeitnehmer wird vor der Ausbeutung durch den wirtschaftlich stärkeren Arbeitgeber geschützt.
		\item Friedensfunktion (Friedenspflicht): Für die Gültigkeitsdauer des Vertrages verpflichten sich die Parteien auf Kampfmaßnahmen zu verzichten. Mit der Friedensfunktion geht die Friedenspflicht einher. Sind erst einmal Tarifverträge geschlossen, verpflichten sich beide Parteien dazu, Frieden zu halten. Auf Seiten des Arbeitgebers besteht die Tariferfüllungspflicht.
		\item Ordnungsfunktion: Alle Arbeitsverträge, die durch den Tarifvertrag erfasst sind, werden gleichartig behandelt.
		\item Richtlinienfunktion: Es werden klare Regelungen für Rechte und Pflichten von Gewerkschaften und Arbeitgeberverbänden getroffen.
	\end{itemize}

\subsubsection{Arten von Tarifverträgen}

	\begin{itemize}
		\item Manteltarifvertrag: Manteltarifverträge legen die allgemeinen Arbeitsbedingungen fest. Dazu zählen Arbeitszeit, Urlaub, Kündigungsfristen, Rationalisierungsschutz, Überstundenzuschläge ...
		\item Rahmentarifvertrag: Rahmentarifverträge regeln beispielsweise die Festlegung von Lohn- und Gehaltsgruppen sowie Merkmale und Tätigkeiten dieser Gruppen.
		\item Lohn-, Gehalts- und Entgelttarifvertrag: Lohn- und Gehaltstarifverträge enthalten Lohnsätze sowie Zu- und Abschläge für einzelne Lohn- und Gehaltsgruppen. Das Tarifgefüge wird dabei von einem sog. Ecklohn bestimmt. Dabei ist der Ecklohn der für einen über 21 Jahre alten Facharbeiter der untersten Tarifgruppe festgesetzte Normallohn, aus dem sich durch prozentuale Zu- und Abschläge die Tariflöhne für andere Lohngruppen berechnen.
		\item Einzeltarifvertrag: Firmentarifvertrag wird mit einem einzelnen Unternehmen geschlossen und auch als Haustarifvertrag bezeichnet. Bei der Unterteilung nach räumlicher Geltung ist der Flächentarifvertrag der häufigste. 
	\end{itemize}

\subsubsection{Geltung von Tarifverträgen}

In einem Tarifvertrag wird unter anderem auch festgelegt, für wen der jeweilige Tarifvertrag gilt. Dies wird festgelegt durch vier Dimensionen: (1) Räumliche Geltung, (2) Fachliche Geltung, (3) Persönliche Geltung und (4) Zeitliche Geltung. Damit werden die Fragen 'Wo gilt der Vertrag?', 'Für welche Branchen?', 'Für welche Gruppen?' und 'Für wie lange?' abgedeckt. Nach Ablauf der zeitlichen Geltung bleibt ein Vertrag bis zum Abschluss eines neuen Vertrages weiterhin gültig (sog. 'Nachhaltigkeit').

Innerhalb des Geltungsbereiches gilt ein Tarifvertrag erst einmal nur für die Gewerkschaftsmitglieder. In der Praxis gewähren Arbeitgeber aber auch nicht organisierten Mitarbeitern dieselben Leistungen wie Gewerkschaftsmitgliedern. 

\subsubsection{Allgemeinverbindlichkeit}

Tarifverträge können vom Bundesministerium für Arbeit und Sozialordnung im Einvernehmen mit dem Tarifausschuss für allgemeien verbindlich erklärt werden. Voraussetzungen dafür sind zum ersten, dass ein Tarifpartner die Allgemeinverbindlichkeit beantragt, zum zweiten, dass der tarifgebundene Arbeitgeber mindestens 50\% der betroffenen Arbeitnehmer beschäftigt und drittens die Allgemeinverbindlichkeit geboten erscheint. 

\subsubsection{Ablauf von Tarifverhandlungen}

\begin{enumerate}
	\item Der alte Vertrag läuft ab oder wird gekündigt.
	\item Der Arbeitgeberverand lehnt Vorderungen ab.
	\item Die Gewerkschaft lehnt das Gegenangebot ab.
	\item Ergebnislose Verhandlungen münden in Warnstreiks.
	\item Zudem wird eine Schlichtungskommission eingesetzt.
	\item Scheitert die Schlichtung kommt es zum Arbeitskampf.
	\item Am Ende des Arbeitskampfes steht ein Kompromis.
\end{enumerate}

\subsubsection{Kampfparität}

Kampfparität bezeichnet das Kräftegleichgewicht zwischen Gewerkschaften und Arbeitgebern. Den Gewerkschaften steht das Mittel 'Streik' zur Verfügung. Arbeitgeber können darauf mit Aussperrung reagieren.

Scheitern erste Tarifverhandlungen und Schlichtungsversuche, kommt es zum  Arbeitskampf und im Zuge dessen zu Streiks. Bei einem Streik handelt es sich um eine planmäßige Arbeitsniederlegung. Dadurch wird dem Unternehmen ein wirtschaftlicher Schaden zugefügt, der den Arbeitgeber dazu veranlassen soll, auf die Forderungen der Gewerkschaft einzugehen. Ein legitimer Streik kann nur von einer Gewerkschaft ausgerufen werden, nachdem die Mitglieder - in der Regel 75\% - im Rahmen einer Urabstimmung für dieses Mittel gestimmt haben. Kommt die Tarifkommission während dieser Zeit zu einer Übereinkunft, wird bei einer weiteren Urabstimmung über die Annahme des neuen Vertrages entschieden; in der Regel müssen mindestens 25\% der Mitglieder zustimmen. Während eines Streikes erhalten die Mitglieder Streikgeld. Als Mitglied einer Gewerkschaft zahlt man Beiträge, die unter anderem dazu genutzt werden, die Streikkasse zu füllen. Im Fall eines Streikes erhalten die Arbeitnehmer von ihrem Arbeitgeber kein Gehalt mehr, sondern von der Gewerkschaft das Streikgeld.

Während des Streikes ruht das Arbeitsverhältnis und die Streikenden sind nicht unfall- oder rentenversichert, jedoch krankenversichert.

Aussperrung ist auf Seiten der Arbeitgeber das Äquivalent zum Streik (s. auch 'Kampfparität'). Stimmt die Mitgleiderversammlung des betroffenen Arbeitgeberverbandes für die Aussperrung, wird allen Arbeitnehmern eines Tarifbezirkes verboten, den Betrieb zu betreten. Das trifft vor allem die nicht organisierten Arbeitnehmer, da diese während der Aussperrung weder Gehalt noch Streikgeld erhalten. Es ist dabei auf die Verhältnismäßigkeit zu achten. Die  Aussperrung hat eine suspendierende Wirkung auf den Arbeitsvertrag. Arbeitnehmer haben nach der Aussperrung einen Anspruch auf Weiterbeschäftigung.

\subsection{[3. Block]}
\subsection{[4. Block]}